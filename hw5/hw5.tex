\documentclass[10pt]{article}
\usepackage[utf8]{inputenc}
\usepackage{amscd}
\usepackage{amsmath}
\usepackage{amssymb}
\usepackage{amsthm}
\usepackage{listings}
\usepackage{enumerate}
\usepackage[all,cmtip]{xy}

\textwidth=15cm \textheight=22cm \topmargin=0.5cm \oddsidemargin=0.5cm \evensidemargin=0.5cm

\newcommand{\sk}{\vskip 10mm}
\newcommand{\bb}[1]{\mathbb{#1}}
\newcommand{\ra}{\rightarrow}
\newcommand{\conj}[1]{\overline{#1}}
\newcommand{\wt}[1]{\widetilde{#1}}
\DeclareMathOperator{\cis}{cis}
\DeclareMathOperator{\res}{Res}
\DeclareMathOperator{\sech}{sech}


\theoremstyle{remark}
\newtheorem{problem}{Problem}
\newtheorem{lemma}{Lemma}[problem]

\theoremstyle{remark}
\newtheorem{tpart}{}[problem]
\newtheorem*{ppart}{}

\begin{document}

\begin{problem}[8.5.7]
  Provide all the details in the proof of the formula for the solution of
  the Dirichlet problem in a strip discussed in Section 1.3. Recall that
  it suffices to compute the solution at the points $z=iy$ with $0<y<1$.
  \begin{enumerate}
  \item[(a)] Show that if $re^{i\theta}=G(iy)$, then
    \[
      re^{i\theta}=i\frac{\cos\pi y}{1+\sin\pi y}
    \]
    This leads to two separate cases: either $0<y\leq 1/2$ and $\theta=\pi/2$
    or $1/2\leq y<1$ and $\theta=-\pi/2$. In either case, show that
    \[
      r^2=\frac{1-\sin\pi y}{1+\sin\pi y}\quad \text{and}\quad P_r(\theta-\varphi)=\frac{\sin\pi y}{1-\cos\pi y\sin\varphi}
    \]
  \item[(b)] In the integral
    $\frac{1}{2\pi}\int_0^\pi P_r(\theta-\varphi)\wt{f}_0(\varphi)d\varphi$
    make the change of variables $t=F(e^{i\varphi})$. Observe that
    \[
      e^{i\varphi}=\frac{i-e^{\pi t}}{i+e^{\pi t}}
    \]
    and then take the imaginary part and differentiate both sides to establish
    the two identities
    \[
      \sin\varphi = \frac{1}{\cosh \pi t}\quad \text{and}\quad \frac{d\varphi}{dt}=\frac{\pi}{\cosh\pi t}
    \]
    Hence deduce that
    \begin{align*}
      \frac{1}{2\pi}\int_0^\pi P_r(\theta-\varphi)\wt{f}_0(\varphi)d\varphi &=\frac{1}{2\pi}\int_0^\pi\frac{\sin\pi y}{1-\cos\pi y\sin\varphi}\wt{f}_0(\varphi)d\varphi\\
                                             &=\frac{\sin\pi y}{2}\int_{-\infty}^\infty\frac{f_0(t)}{\cosh\pi t-\cos\pi y}dt
    \end{align*}
  \item[(c)] Use a similar argument to prove the formula for the integral
    $\frac{1}{2\pi}\int_{-\pi}^0 P_r(\theta-\varphi)\wt{f}_1(\varphi)d\varphi$.
  \end{enumerate}
\end{problem}

\begin{proof}
  \begin{enumerate}
  \item[(a)] By the definition of $G$ we start with
    \[
      re^{i\theta}=\frac{i-e^{i\pi y}}{i+e^{i\pi y}}
    \]
    Replace the $e^{i\theta}$ with sines and cosines, followed by multiplying the top and
    bottom by the conjugate of the bottom to get
    \[
      \frac{i-e^{i\pi y}}{i+e^{i\pi y}}=\frac{i-\cos\pi y-i\sin\pi y}{i+\cos\pi y+i\sin\pi y}\cdot\frac{\cos\pi y-i(1+\sin\pi y)}{\cos\pi y-i(1+\sin\pi y)}
    \]
    Which then simplifies to
    \[
      i\frac{\cos\pi y}{1+\sin\pi y}
    \]
    In the first case listed above we lie on the positive imaginary axis, and in the second case listed
    above we are on the negative imaginary axis. Either way we have
    \begin{align*}
      (\pm i r)=-r^2&= -\frac{\cos^2\pi y}{(1+\sin\pi y)^2}\\
      r^2 &= \frac{1-\sin^2\pi y}{(1+\sin\pi y)^2}\\
                 &= \frac{1-\sin\pi y}{1+\sin \pi y}
    \end{align*}
    By definition of $P_r(\theta-\varphi)$ we have
    \[
      P_r(\theta-\varphi)=\frac{1-r^2}{1-2r\cos(\theta-\varphi)+r^2}
    \]
    Substitute in for $r^2$ and simplify to get
    \begin{align*}
      P_r(\theta-\varphi)&=\frac{1-r^2}{1-2r\cos(\theta-\varphi)+r^2}\\
             &= \frac{1-\frac{1-\sin\pi y}{1+\sin \pi y}}{1-2r\cos(\theta-\varphi)+\frac{1-\sin\pi y}{1+\sin \pi y}}\\
             &=\frac{1-\frac{1-\sin\pi y}{1+\sin \pi y}}{1-2r(\cos\theta\cos\varphi+\sin\theta\sin\varphi)+\frac{1-\sin\pi y}{1+\sin \pi y}}\cdot\frac{1+\sin\pi y}{1+\sin\pi y}\\
             &=\frac{2\sin\pi y}{2-2r(1+\sin\pi y)(\sin\theta\sin\varphi)}\\
             &=\frac{\sin\pi y}{1-(e^{i(\theta+\pi/2)}\cos\pi y)(\sin\theta\sin\varphi)}
    \end{align*}
    From here since we know what $\theta$ is, we can simplify to
    \[
      P_r(\theta-\varphi)=\frac{\sin\pi y}{1-\cos\pi y\sin\varphi}
    \]
  \item[(b)]
  \item[(c)]
  \end{enumerate}
\end{proof}

\sk

\begin{problem}[8.5.9]
  Prove that the function $u$ defined by
  \[
    u(x,y)=\Re(\frac{i+z}{i-z}),\quad u(0,1)=0
  \]
  is harmonic in the unit disc and vanishes on the boundary. Note that
  $u$ is not bounded in $\bb{D}$.
\end{problem}

\begin{proof}
  Since $u$ is the real part of a holomorphic function it is harmonic. If
  we write $z=x+iy$ we can rewrite $u(x,y)$ as
  \[
    \Re(\frac{i+z}{i-z})=\frac{1-x^2-y^2}{x^2+(1-y)^2}=\frac{1-|z|^2}{x^2+(1-y)^2}
  \]
  If $|z|=1$ and $z\neq i$, the top vanishes giving us that $u(x,y)=0$.
  Since we decreed that $u(0,1)=0$ we can conclude that $u$ vanishes
  on $\partial\bb{D}$.
\end{proof}

\sk

\begin{problem}[8.5.16]
  Let
  \[
    f(z)=\frac{i-z}{i+z}\quad\text{and}\quad f^{-1}(w)=i\frac{1-w}{1+w}
  \]
  \begin{enumerate}
  \item[(a)] Given $\theta\in\bb{R}$, find real numbers $a,b,c,d$ so that
    $ad-bc=1$, and so that for any $z\in\bb{H}$
    \[
      \frac{az+b}{cz+d}=f^{-1}(e^{i\theta}f(z))
    \]
  \item[(b)] Given $\theta\in\bb{R}$, find real numbers $a,b,c,d$ so that
    $ad-bc=1$, and so that for any $z\in\bb{H}$
    \[
      \frac{az+b}{cz+d}=f^{-1}(\psi_\alpha(f(z)))
    \]
    with $\psi_a$ defined in Section 2.1.
  \item[(c)] Prove that if $g$ is an automorphism of the unit disc, then there
    exist real numbers $a,b,c,d$ such that $ad-bc=1$ and so that for any
    $z\in\bb{H}$
    \[
      \frac{az+b}{cz+d}=f^{-1}\circ g\circ f(z)
    \]
    [Hint: Use parts (a) and (b)].
  \end{enumerate}
\end{problem}

\begin{proof}
  \begin{enumerate}
  \item[(a)] The equivalent transformation is
    \[
      \frac{\sin(\theta/2)+\cos(\theta/2)z}{\cos(\theta/2)-\sin(\theta/2)z}
    \]
    To see this use begin $f^{-1}(e^{i\theta}f(z))$ and simplify to get
    \begin{align*}
      \frac{(e^{i\theta}-1)+iz(1+e^{i\theta})}{i(1+e^{i\theta})-(e^{i\theta}-1)z}&=
    \end{align*}
  \item[(b)] Similarly the equivalent transformation is
    \[
      \frac{\Re(\alpha)-1+z\Im(\alpha)}{z(\Re(\alpha)+1)-\Im(\alpha)}\cdot\frac{(1-|\alpha|^2)^{-1/2}}{(1-|\alpha|^2)^{-1/2}}
    \]
    To derive this, begin with $f^{-1}\circ\psi_\alpha\circ f(z)$ to get
    \begin{align*}
      \frac{i-i\frac{\alpha-\frac{i-z}{i+z}}{1-\conj{\alpha}\frac{i-z}{i+z}}}{1+\frac{\alpha-\frac{i-z}{i+z}}{1-\conj{\alpha}\frac{i-z}{i+z}}}&=\\
    \end{align*}
  \item[(c)] From class we know that any automorphism of $\bb{D}$ is the
    composition of a rotation $\rho(z)$ and a $\psi_\alpha(z)$. Let $g(z)=\rho\circ\psi_\alpha(z)$.
    Then
    \begin{align*}
      f^{-1}\circ g\circ f(z) &= f^{-1}\circ\rho\circ\psi_\alpha \circ f(z)\\
                      &= (f^{-1}\circ\rho\circ f)\circ (f^{-1}\circ\psi_\alpha \circ f)(z)\\
    \end{align*}
    At this point we know that both $f^{-1}\circ\rho\circ f$ and $f^{-1}\circ\psi_\alpha \circ f$
    are of the form $\frac{az+b}{cz+d}$ with $ad-bc=1$. Moreover since
    composition of two transformations of the above form gives another of
    its kind it must be that $g(z)$ is of the form $\frac{az+b}{cz+d}$.
  \end{enumerate}
\end{proof}

\sk

\begin{problem}[8.5.20]
  Other examples of elliptic integrals providing conformal maps form the upper
  half-plane to rectangles providing conformal maps from the upper half-plane
  to rectangles are given below.
  \begin{enumerate}
  \item[(a)] The function
    \[
      S(z)=\int_0^z\frac{d\zeta}{\sqrt{\zeta(\zeta-1)(\zeta-\lambda)}},\quad \lambda\in\bb{R}\setminus\{1\}
    \]
    maps the upper half-plane conformally to a rectangle, one of whose vertices
    is the image of the point at infinity.
  \item[(b)] In the case $\lambda=-1$, the image of
    \[
      S(z)=\int_0^z\frac{d\zeta}{\sqrt{\zeta(\zeta^2-1)}}
    \]
    is a square whose side lengths are $\frac{\Gamma^2(1/4)}{2\sqrt{2\pi}}$.
  \end{enumerate}
\end{problem}

\begin{proof}
  \begin{enumerate}
  \item[(a)] By proposition 8.4.1, we fall into the second case, and so the
    image of the upper half plane is a quadrangle with corners $S(0),S(1),S(\lambda)$, and
    $S(\infty)$ where the angle corresponding to $S(\infty)$ is $\pi/2$. In addition we also get
    from the proposition that the rest of the angles are $\pi/2$ since they are each to
    the $1/2$ power. So the polygon must in fact be a rectangle.
  \item[(b)] From above we know that the shape is a rectangle. All that is left
    to confirm is that the side lengths are $\frac{\Gamma^2(1/4)}{2\sqrt{2\pi}}$.
    Starting with the side $S(0)$ to $S(1)$ make the substitution $\sqrt{u}=\zeta$ to get
    \[
      \int_0^1\frac{d\zeta}{\sqrt{\zeta(\zeta^2-1)}}=\frac{1}{2}\int_0^1u^{-3/4}(1-u)^{-1/2}du = \frac{B(1/4,1/2)}{2}
    \]
    From the last homework we have that
    \[
      \frac{B(1/4,1/2)}{2}=\frac{\Gamma(1/4)\Gamma(1/2)}{2\Gamma(3/4)}=\frac{\Gamma(1/4)\sqrt{\pi}}{2\frac{\sqrt{2}\pi}{\Gamma(1/4)}}=\frac{\Gamma^2(1/4)}{2\sqrt{2\pi}}
    \]
    To get another side we do $|S(-1)-0|=|S(-1)|$ instead doing the substitution $-\sqrt{u}=\zeta$.
  \end{enumerate}
\end{proof}

\sk

\begin{problem}[8.5.21]
  We consider the conformal mappings to triangles.
  \begin{enumerate}
  \item[(a)] Show that
    \[
      S(w)=\int_0^w z^{-\beta_1}(1-z)^{-\beta_2}dz
    \]
    with $0<\beta_1<1$, $0<\beta_2<$, and $1<\beta_1+\beta_2<2$, maps $\bb{H}$
    to a triangle whose vertices are the images of $0,1,$ and
    $\infty$, and with angles $\alpha_1\pi,\alpha_2\pi$, and $\alpha_3\pi$,
    where $\alpha_j+\beta_j=1$ and $\beta_1+\beta_2+\beta_3=2$.
  \item[(b)] What happens when $\beta_1+\beta_2=1$?
  \item[(c)] What happens when $0<\beta_1+\beta_2<1$?
  \item[(d)] In (a), the length of the side of the triangle opposite angle
    $\alpha_j\pi$ is
    $\frac{\sin(\alpha_j\pi)}{\pi}\Gamma(\alpha_1)\Gamma(\alpha_2)\Gamma(\alpha_3)$.
  \end{enumerate}
\end{problem}

\begin{proof}
  \begin{enumerate}
  \item[(a)] Using proposition 8.4.1, we have that the image $S(\bb{H})$ is a
    is a triangle with points at $S(0),S(1)$, and $S(\infty)$. The angles corresponding
    to each side are then $\alpha_0\pi,\alpha_1\pi,\alpha_\infty\pi$ where $\alpha_z=1-\beta_z$ and $\beta_0+\beta_1+\beta_\infty=2$ from
    the proposition.
  \item[(b)] Following in a similar fashion to the last problem we
    get a ``degenerate'' triangle that is a straight line (angle at infinity is $\pi$).
  \item[(c)] We get a triangle with a vertex at infinity and a vertex at $S(0),S(1)$ with
    non-intersecting rays emerging from each.
  \item[(d)] If we have one side the others follow from using law of sines. The length
    of the side corresponding to $S(0)$ and $S(1)$ has length $|S(1)-S(0)|=|S(1)|=|\int_0^2 z^{-\beta_1}(1-z)^{-\beta_2}dz|$.
  \end{enumerate}
\end{proof}

\sk

\begin{problem}[8.6.2]
  The angle between two non-zero complex numbers $z$ and $w$ (taken in that
  order) is simply the oriented angle, in $(-\pi,\pi]$, that is formed between
  the two vectors in $\bb{R}^2$ corresponding to the points $z$ and $w$. This
  oriented angle, say $\alpha$, is uniquely determined by the two quantities
  \[
    \frac{(z,w)}{|z||w|}\quad\text{and}\quad \frac{(z,-iw)}{|z||w|}
  \]
  which are simply the cosine and sine of $\alpha$, respectively. Here, the
  notation $(\cdot,\cdot)$ corresponds to the usual Euclidean inner product in
  $\bb{R}^2$, which in terms of complex numbers takes the form
  $(z,w)=\Re(z\conj{w})$.

  In particular, we may now consider two smooth curves
  $\gamma:[a,b]\rightarrow\bb{C}$ and $\eta:[a,b]\rightarrow\bb{C}$, that
  intersect at $z_0$, say $\gamma(t_0)=\eta(t_0)=z_0$ for some
  $t_0\in(a,b)$. If the quantities $\gamma'(t_0)$ and $\eta'(t_0)$ are
  non-zero, then they represent the tangents to the curves $\gamma$ and
  $\eta$ at the point $z_0$, and we say that the two curves intersect at
  $z_0$ at the angle formed by the two vectors $\gamma'(t_0)$ and $\eta'(t_0)$.

  A holomorphic function $f$ defined near $z_0$ is said to
  \textbf{preserve angles} at $z_0$ if for any two smooth curves $\gamma$ and
  $\eta$ intersecting at $z_0$, the angle formed between the curves $\gamma$ and
  $\eta$ at $z_0$ equals the angle formed between the curves $f\circ\gamma$
  and $f\circ\eta$ at $f(z_0)$. In particular we assume that the tangents to
  the curves $\gamma,\eta,f\circ\gamma$, and $f\circ\eta$ at the point $z_0$ and
  $f(z_0)$ are all non-zero.
  \begin{enumerate}
  \item[(a)] Prove that if $f:\Omega\rightarrow\bb{C}$ is holomorphic, and
    $f'(z_0)\neq 0$, then $f$ preserves angles at $z_0$. [Hint: Observe that
    \[
      (f'(z_0)\gamma'(t_0),f'(z_0)\eta'(t_0))=|f'(z_0)|^2(\gamma'(t_0),\eta'(t_0)))
    \]
      ]
    \item[(b)] Conversely, prove the following: suppose
      $f:\Omega\rightarrow\bb{C}$ is complex-valued function, that is real
      differentiable at $z_0\in\Omega$, and $J_f(z_0)\neq 0$. If $f$ preserves
      angles at $z_0$, then $f$ is holomorphic at $z_0$ with $f'(z_0)\neq 0$.
  \end{enumerate}
\end{problem}

\begin{proof}
  \begin{enumerate}
  \item[(a)] From the problem description if the two listed quantities are
    preserved then so is the angle. Using the suggested inequality along
    with the chain rule we get
    \begin{align*}
      \frac{(f'(z_0)\gamma'(t_0),f'(z_0)\eta'(t_0))}{|f'(z_0)\gamma'(t_0)||f'(z_0)\eta'(t_0)|}&=\frac{|f'(z_0)|^2(\gamma'(t_0),\eta'(t_0)))}{|f'(z_0)\gamma'(t_0)||f'(z_0)\eta'(t_0)|}\\
      &= \frac{(\gamma'(t_0),\eta'(t_0)))}{|\gamma'(t_0)||\eta'(t_0)|}\\
    \end{align*}
    And similarly
    \begin{align*}
      \frac{(f'(z_0)\gamma'(t_0),if'(z_0)\eta'(t_0))}{|f'(z_0)\gamma'(t_0)||f'(z_0)\eta'(t_0)|}&=\frac{|f'(z_0)|^2(\gamma'(t_0),i\eta'(t_0)))}{|f'(z_0)\gamma'(t_0)||f'(z_0)\eta'(t_0)|}\\
      &= \frac{(\gamma'(t_0),i\eta'(t_0)))}{|\gamma'(t_0)||\eta'(t_0)|}\\
    \end{align*}
  \item[(b)] Since angles are preserved by $f$ we can coerce the Cauchy-Riemann
    equations out of the two quantities above with $z(t)$ and $w(t)=f\circ z(t)$
    and $z(t)$ with $\conj{w(t)}$. Since they correlate to sine and cosine this
    will give us
    \[
      (z,-iw)=0,\quad (z,\conj{w})=0
    \]
    \textbf{That's the plan anyways.}
  \end{enumerate}
\end{proof}

\sk

%%%%%%%%%%%%%%%%%%%%%%%%%%%%%%%%%%%%%%%%%%%%%%%%%%%%%%%%%%%%%%%%%%%%%%%%%%%%%
\end{document}
