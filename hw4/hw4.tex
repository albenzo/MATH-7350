\documentclass[10pt]{article}
\usepackage[utf8]{inputenc}
\usepackage{amscd}
\usepackage{amsmath}
\usepackage{amssymb}
\usepackage{amsthm}
\usepackage{listings}
\usepackage{enumerate}
\usepackage[all,cmtip]{xy}

\textwidth=15cm \textheight=22cm \topmargin=0.5cm \oddsidemargin=0.5cm \evensidemargin=0.5cm

\newcommand{\sk}{\vskip 10mm}
\newcommand{\bb}[1]{\mathbb{#1}}
\newcommand{\ra}{\rightarrow}
\newcommand{\conj}[1]{\overline{#1}}
\newcommand{\wt}[1]{\widetilde{#1}}
\DeclareMathOperator{\cis}{cis}
\DeclareMathOperator{\res}{Res}
\DeclareMathOperator{\sech}{sech}


\theoremstyle{plain}
\newtheorem{problem}{Problem}
\newtheorem{lemma}{Lemma}[problem]

\theoremstyle{remark}
\newtheorem{tpart}{}[problem]
\newtheorem*{ppart}{}

\begin{document}

\begin{problem}[5.6.2]
  Find the order of growth of the following entire functions:
  \begin{enumerate}
  \item[(a)] $p(z)$ where $p$ is a polynomial.
  \item[(b)] $e^{bz^n}$ for $b\neq 0$.
  \item[(c)] $e^{e^z}$.
  \end{enumerate}
\end{problem}

\begin{proof}
  \begin{enumerate}
  \item[(a)] Let $n=\deg p(z)$ and $\rho>0$. Choose $C>|a_0|$ such that $|p(z)|\leq C|z|^n$
    and $m$ so that $\rho m>n$. Then
    \[
      m!Ce^{|z|^\rho}=m!C\sum_{k=0}^\infty\frac{|z|^{\rho k}}{k!}\geq C|z|^{\rho m}\geq C|z|^n\geq|p(z)|
    \]
    Since this holds for any $\rho>0$ we have that $\rho_{p(z)}=0$.
  \item[(b)] Using the Taylor expansion we get
    \[
      |e^{bz^n}|\leq|\sum_{m=0}^\infty\frac{(z^n)^m}{m!}|\leq\sum_{m=0}^\infty|\frac{(z^n)^m}{m!}|\leq\sum_{m=0}^\infty\frac{|z|^{nm}}{m!}\leq e^{b|z|^n}
    \]
    which shows that $\rho_{e^{bz^n}}\leq n$. However if we choose the exponent in the
    definition of order to be $b$ we get exactly $e^{bz^n}=e^{Bz^n}$. From this
    we can conclude that the order of $e^{bz^n}$ is exactly $n$.

  \item[(c)] For $e^{e^z}$ suppose that it had order $n$. Then if $z=x+iy$
    \[
      |e^{e^z}|=|e^{e^x(\cos x+i\sin y)}|=e^{e^x\cos x}< Ae^{B|z|^n}
    \]
    Take the logarithm of both sides to get
    \[
      e^x\cos x < C|z|^n
    \]
    which is most assuredly not true for all $z\in\bb{C}$. As
    such $e^{e^z}$ does not have finite order.
  \end{enumerate}
\end{proof}

\sk

\begin{problem}[5.6.6]
  Prove Wallis's product formula
  \[
    \frac{\pi}{2}=\prod_{m=1}^\infty\frac{(2m)^2}{(2m-1)(2m+1)}
  \]
  [Hint: Use the product formula for $\sin z$ at $z=\pi/2$.]
\end{problem}

\begin{proof}
  Plugging in $z=\pi/2$ we get
  \[
    1=\frac{\pi}{2}\prod_{n=1}^\infty(1-\frac{1}{4n^2})=\frac{\pi}{2}\prod_{n=1}^\infty\frac{(2n+1)(2n-1)}{(2n)^2}
  \]
  Which then dividing by the product gives us
  \[
    \prod_{n=1}^\infty\frac{(2n)^2}{(2n+1)(2n-1)}=\frac{\pi}{2}
  \]
  as desired.
\end{proof}

\sk

\begin{problem}[5.6.8]
  Prove that for every $z$ the product below converges, and
  \[
    \prod_{k=1}^\infty \cos(z/2^k)=\frac{\sin z}{z}
  \]
  [Hint: Use the fact that $\sin 2z=2\sin z\cos z$.]
\end{problem}

\begin{proof}
  Start with using the suggested identity with $z/2$ to get
  \[
    \sin z=2\sin(z/2)\cos(z/2)
  \]
  Then iterate usage of the identity and divide by $z$ to get
  \[
    \frac{\sin z}{z} = \frac{2^N\sin(z/2^N)}{z}\prod_{k=1}^N\cos(z/2^k)
  \]
  Use Wallis's formula on the right term to get
  \[
    \frac{2^N\sin(z/2^N)}{z}=\frac{2^N}{z}\cdot\frac{z}{2^N}\prod_{k=1}^\infty(1-\frac{(\frac{z}{2^N})^2}{k^2\pi^2})=\prod_{k=1}^\infty(1-\frac{(\frac{z}{2^N})^2}{k^2\pi^2})
  \]
  which goes to $1$ as $N\rightarrow\infty$. Which in turn gives us that
  \[
    \frac{\sin z}{z}=\prod_{k=1}^\infty\cos(z/2^k)
  \]
  as desired.
\end{proof}

\sk

\begin{problem}[5.6.10(b)]
  Show that the Hadamard product for $\cos z$ is 
  \[
    \cos\pi z=\prod_{n=0}^\infty 1-\frac{4z^2}{(2n+1)^2}
  \]
\end{problem}

\begin{proof}
  First note that $\cos z$ has growth order $1$. As such for the Hadamard
  product formula we get
  \begin{align*}
    \cos\pi z&=e^{P(z)}z^0\prod_{n=-\infty}^\infty E_1(2z/(2n+1))\\
           &=e^{P(z)}\prod_{n=-\infty}^\infty (1-\frac{2z}{2n+1})e^{2z/(2n+1)}\\
           &=e^{P(z)}(\prod_{n=0}^\infty (1-\frac{2z}{2n+1})e^{2z/(2n+1)})(\prod_{n=0}^\infty (1+\frac{2z}{2n+1})e^{-2z/(2n+1)})\\
           &=e^{P(z)}\prod_{n=0}^\infty(1-\frac{4z^2}{(2n+1)^2})
  \end{align*}
  Now we must show that $P(z)=0$. Since $P$ is at most degree $1$
  it will be sufficient to show $P$ evaluates to zero at two distinct
  values for $z$. Begin with $z=0$
  \[
    1=e^{P(0)}\prod_{n=0}^\infty 1
  \]
  Giving that $P(0)=0$. Then use $z=2$ to get
  \[
    1=e^{P(2)}\prod_{n=0}^\infty(1-\frac{16}{(2n+1)^2})
  \]
  The partial product for the right term is
  \[
    \prod_{n=0}^N1-\frac{16}{(2n+1)^2}=\frac{(2N+3)(2N+5)}{4N^2-1}
  \]
  Which implies that $\prod_{n=0}^\infty 1-\frac{16}{(2n+1)^2}=1$ and as
  such $P(2)=0$ as well. Thus $P(z)=0$ and we have the formula
  \[
    \cos\pi z=\prod_{n=0}^\infty 1-\frac{4z^2}{(2n+1)^2}
  \]
\end{proof}

\sk

\begin{problem}[6.3.5]
  Use the fact that $\Gamma(s)\Gamma(1-s)=\pi/\sin\pi s$ to prove that
  \[
    |\Gamma(1/2+it)|=\sqrt{\frac{2\pi}{e^{\pi t}+e^{-\pi t}}}=\sqrt{\pi\sech{\pi t}}
  \]
  whenever $t\in \bb{R}$.
\end{problem}

\begin{proof}
  Use the above identity with $s=1/2+it$ to get
  \begin{align*}
    \Gamma(1/2+it)\Gamma(1/2-it) &= \frac{2\pi}{\sin\pi s}\\
                       &= \frac{2\pi i}{e^{i\pi/2-\pi t}-e^{-i\pi/2+\pi t}}\\
                       &= \frac{2\pi i}{i(e^{\pi t}-e^{-\pi t})}\\
                       &= \frac{2\pi}{e^{\pi t}-e^{-\pi t}}\\
  \end{align*}
  Then using the fact that $\Gamma(\bar{s})=\overline{\Gamma(s)}$ and
  that $s\bar{s}=|s|^2$ we get
  \[
    |\Gamma(1/2+it)|^2=\frac{2\pi}{e^{\pi t}-e^{-\pi t}}
  \]
  Which leads us to the desired identity
  \[
    |\Gamma(1/2+it)|=\sqrt{\frac{2\pi}{e^{\pi t}-e^{-\pi t}}}
  \]
\end{proof}

\sk

\begin{problem}[6.3.7]
  The \textbf{Beta function} is defined for $\Re(\alpha)>0$ and
  $\Re(\beta)>0$ by
  \[
    B(\alpha,\beta)=\int_0^1(1-t)^{\alpha-1}t^{\beta-1}dt
  \]
  \begin{enumerate}
  \item[(a)] Prove that
    $B(\alpha,\beta)=\frac{\Gamma(\alpha)\Gamma(\beta)}{\Gamma(\alpha+\beta)}$.
  \item[(b)] Show that
    $B(\alpha,\beta)=\int_0^\infty\frac{u^{\alpha-1}}{(1+u)^{\alpha+\beta}}du$.
  \end{enumerate}
  [Hint: For part (a), note that
  \[
    \Gamma(\alpha)\Gamma(\beta)=\int_0^\infty\int_0^\infty t^{\alpha-1}s^{\beta-1}e^{-t-s}dtds
  \]
  and make the change of variables $s=ur,t=u(1-r)$.]
\end{problem}

\begin{proof}
  \begin{enumerate}
  \item[(a)] Begin with $\Gamma(\alpha)\Gamma(\beta)$ and make the above suggested
    substitution to get
    \begin{align*}
      \Gamma(\alpha)\Gamma(\beta) &= (\int_0^\infty e^{-t}t^{\alpha-1}dt)(\int_0^\infty e^{-s}s^{\beta-1}ds)\\
               &=\int_0^\infty\int_0^\infty t^{\alpha-1}s^{\beta-1}e^{-(t+s)}dtds\\
               &=\int_0^1\int_0^\infty(u(1-r))^{\alpha-1}(ur)^{\beta-1}e^{-u}(-u)dudr\\
               &=\int_0^1\int_0^\infty-u^{\alpha+\beta-1}e^{-u}(1-r)^{\alpha-1}r^{\beta-1}dudr\\
               &=\int_0^1(\int_0^\infty-u^{\alpha+\beta-1}e^{-u}du)(1-r)^{\alpha-1}r^{\beta-1}dr\\
               &=\int_0^1\Gamma(\alpha+\beta)(1-r)^{\alpha-1}r^{\beta-1}dr\\
               &=\Gamma(\alpha+\beta)\int_0^1(1-r)^{\alpha-1}r^{\beta-1}dr\\
               &=\Gamma(\alpha+\beta)B(\alpha,\beta)\\
    \end{align*}
    Which when we divide through gives us
    \[
      B(\alpha,\beta)=\frac{\Gamma(\alpha)\Gamma(\beta)}{\Gamma(\alpha+\beta)}
    \]
  \item[(b)] Begin with the integral side of the problem and make the
    substitution $\frac{1-t}{t}=u$ to get
    \begin{align*}
      \int_0^\infty\frac{u^{\alpha-1}}{(1+u)^{\alpha+\beta}}du &= \int_0^\infty (\frac{u}{1+u})^{\alpha-1}(1+u)^{-\beta-1}du\\
                                                  &=\int_1^0 (1-t)^{\alpha-1}(t^{-1})^{-\beta-1}(-t^{-2})dt\\
                                                  &=\int_0^1 (1-t)^{\alpha-1}t^{\beta-1}dt\\
                                                  &=B(\alpha,\beta)
    \end{align*}
  \end{enumerate}
\end{proof}

\sk

\begin{problem}[6.3.10]
  An integral of the form
  \[
    F(z)=\int_0^\infty f(t)t^{z-1}dt
  \]
  is called a \textbf{Mellin transform}, and we shall write
  ${\cal M}(f)(z)=F(z)$. For example, the gamma function is the Mellin
  transform of the function $e^{-t}$.
  \begin{enumerate}
  \item[(a)] Prove that
    \[
      {\cal M}(\cos)(z)=\int_0^\infty\cos(t)t^{z-1}dt=\Gamma(z)\cos(\frac{\pi z}{2})
    \]
    for $0<\Re(z)<1$ and
    \[
      {\cal M}(\sin)(z)=\int_0^\infty\sin(t)t^{z-1}dt=\Gamma(z)\sin(\frac{\pi z}{2})
    \]
    for $0<\Re(z)<1$.
  \item[(b)] Show that the second of the above is valid in the larger strip
    $-1<\Re(z)<1$, and that as a consequence, one has
    \[
      \int_0^\infty \frac{\sin x}{x}dx=\frac{\pi}{2}\quad \text{and}\quad \int_0^\infty\frac{\sin x}{x^{3/2}}dx=\sqrt{2\pi}
    \]
  \end{enumerate}
  [Hint: For the first part, consider the integral of the function
  $f(w)=e^{-w}w^{z-1}$ around the quarter annulus. Use the analytic continuation
  to prove the second part.]
\end{problem}

\begin{proof}
  \begin{enumerate}
  \item[(a)] Let $\gamma_{\epsilon,R}$ be the quarter annulus in the upper right quarter
    plane with inner radius $\epsilon$ and outer radius $R$. Since there are no
    singularities within for $f(w)$, the value of the integral is zero. Thus we have
    \[
      \int_{\gamma_{\epsilon,R}}e^{-w}w^{z-1}dw = 0 = \int_\epsilon^R e^{-t}t^{z-1}dt + \int_\epsilon^R e^{-it}t^{z-1}idt+\int_{C_\epsilon}e^{-w}w^{z-1}dw+\int_{C_R}e^{-w}w^{z-1}dw
    \]
    The leftmost term in the sum is $\Gamma(z)$, the two integrals about $C_\epsilon$ and $C_R$
    will both approach zero as $\epsilon\rightarrow 0$ and $R\rightarrow \infty$. This gives us
    \[
      0 = \Gamma(z)+i^z\int_0^\infty e^{-it}t^{z-1}dt=\Gamma(z)+i^z(\int_0^\infty\cos(t)t^{z-1}dt-i\int_0^\infty\sin(t)t^{z-1}dt)=\Gamma(z)+i^z({\cal M}(\cos)(z)-i{\cal M}(\sin)(z))
    \]
    Rearrange to get
    \[
      {\cal M}(\cos)(z)-i{\cal M}(\sin)(z)= \Gamma(z)i^{-z}=\Gamma(z)\cos(\pi z/2)-i\Gamma(z)\sin(\pi z/2)
    \]
    If $z$ is real then by taking the real and imaginary parts of the above
    equality we get
    \[
      {\cal M}(\cos)(z)=\int_0^\infty\cos(t)t^{z-1}dt=\Gamma(z)\cos(\frac{\pi z}{2})
    \]
    and
    \[
      {\cal M}(\sin)(z)=\int_0^\infty\sin(t)t^{z-1}dt=\Gamma(z)\sin(\frac{\pi z}{2})
    \]

    
    For ${\cal M}(\cos)(z)$ the original integral exists in the strip defined
    by $0<\Re(z)<1$. Since the integral agrees with $\Gamma(z)\cos(\pi z/2)$ on
    interval $(0,1)$ and $\Gamma(z)\cos(\pi z/2)$ is analytic in said strip it
    must be that ${\cal M}(\cos)(z)$ continues analytically to the whole
    strip with values $\Gamma(z)\cos(\pi z/2)$. The same reasoning applies
    to ${\cal M}(\sin)(z)$.
  \item[(b)] Using that $\Gamma(z)=(z-1)\Gamma(z-1)$ and that $\cos(z)=\sin(z+\pi/2)$
    we can continue ${\cal M}(\sin)(z)$ to the larger strip by
    \[
      {\cal M}(\sin)(z-1)=\Gamma(z-1)\sin(\pi/2(z-1))=\frac{\Gamma(z)\cos(\pi z/2)}{z-1}
    \]
    Now if we plug-in $-1/2$ in for $z$ we get
    \[
      {\cal M}(\sin)(-1/2)=\int_0^\infty \frac{\sin(t)}{t^{3/2}}dt = \frac{\Gamma(1/2)\cos(-\pi/4)}{-1/2}=\frac{2\sqrt{\pi}}{\sqrt{2}}=\sqrt{2\pi}
    \]
    Using $z=0$ we have
    \begin{align*}
      {\cal M}(\sin)(0)=\int_0^\infty\frac{\sin(t)}{t}dt &= \lim_{z\rightarrow 0}\Gamma(z)\sin(\pi z/2)\\
                                                 &= \lim_{z\rightarrow 0}\Gamma(z+1)\frac{\sin(\pi z/2)}{z}\\
                                                 &=(\lim_{z\rightarrow 0}\Gamma(z+1))(\lim_{z\rightarrow 0}\frac{\sin(\pi z/2)}{z})\\
                                                 &=1\lim_{z\rightarrow 0}\frac{\sin(\pi z/2)}{z}\\
                                                 &=\lim_{z\rightarrow 0}\frac{\pi/2\cos(\pi z/2)}{1}\\
                                                 &=\pi/2
    \end{align*}
  \end{enumerate}
\end{proof}

\sk

\begin{problem}[6.3.15]
  Prove that for $\Re(s)>1$,
  \[
    \zeta(s)=\frac{1}{\Gamma(s)}\int_0^\infty\frac{x^{s-1}}{e^x-1}dx
  \]
  [Hint: Write $1/(e^x-1)=\sum_{n=1}^\infty e^{-nx}$.]
\end{problem}

\begin{proof}
  Start with the right side of the equation and use the suggested series
  with a substitution $t=nx$ to get
  \begin{align*}
    \frac{1}{\Gamma(s)}\int_0^\infty \frac{x^{s-1}}{e^x-1}dx &= \frac{1}{\Gamma(s)}\int_0^\infty\sum_{n=1}^\infty x^{s-1}e^{-nx}dx\\
                                                &=\frac{1}{\Gamma(s)}\sum_{n=1}^\infty \int_0^\infty e^{-nx}dx\\
                                                &=\frac{1}{\Gamma(s)}\sum_{n=1}^\infty \int_0^\infty n^{-s}t^{s-1}e^{-t}dt\\
                                                &=\frac{1}{\Gamma(s)}\sum_{n=1}^\infty n^{-s}\int_0^\infty t^{s-1}e^{-t}dt\\
                                                &=\frac{1}{\Gamma(s)}\sum_{n=1}^\infty n^{-s}\Gamma(s)\\
                                                &=\sum_{n=1}^\infty n^{-s}\\
                                                &= \zeta(s)
  \end{align*}
\end{proof}

%%%%%%%%%%%%%%%%%%%%%%%%%%%%%%%%%%%%%%%%%%%%%%%%%%%%%%%%%%%%%%%%%%%%%%%%%%%%%
\end{document}
