\documentclass[10pt]{article}
\usepackage[utf8]{inputenc}
\usepackage{amscd}
\usepackage{amsmath}
\usepackage{amssymb}
\usepackage{amsthm}
\usepackage{listings}
\usepackage{enumerate}
\usepackage[all,cmtip]{xy}

\textwidth=15cm \textheight=22cm \topmargin=0.5cm \oddsidemargin=0.5cm \evensidemargin=0.5cm

\newcommand{\sk}{\vskip 6mm}
\newcommand{\bb}[1]{\mathbb{#1}}
\newcommand{\ra}{\rightarrow}
\newcommand{\conj}[1]{\overline{#1}}
\newcommand{\wt}[1]{\widetilde{#1}}
\DeclareMathOperator{\cis}{cis}
\DeclareMathOperator{\res}{Res}
\DeclareMathOperator{\sech}{sech}


\theoremstyle{plain}
\newtheorem{problem}{Problem}
\newtheorem{lemma}{Lemma}[problem]

\theoremstyle{remark}
\newtheorem{tpart}{}[problem]
\newtheorem*{ppart}{}

\begin{document}

\begin{problem}[5.6.2]
  Find the order of growth of the following entire functions:
  \begin{enumerate}
  \item[(a)] $p\left(z\right)$ where $p$ is a polynomial.
  \item[(b)] $e^{bz^n}$ for $b\neq 0$.
  \item[(c)] $e^{e^z}$.
  \end{enumerate}
\end{problem}

\begin{proof}
  \begin{enumerate}
  \item[(a)] Let $n=\deg p\left(z\right)$ and $\rho>0$. Choose $C>|a_0|$ such that $|p\left(z\right)|\leq C|z|^n$
    and $m$ so that $\rho m>n$. Then
    \[
      m!Ce^{\left|z\right|^\rho}=m!C\sum_{k=0}^\infty\frac{\left|z\right|^{\rho k}}{k!}\geq C\left|z\right|^{\rho m}\geq C\left|z\right|^n\geq\left|p\left(z\right)\right|
    \]
    Since this holds for any $\rho>0$ we have that $\rho_{p\left(z\right)}=0$.
  \item[(b)] Using the Taylor expansion we get
    \[
      \left|e^{bz^n}\right|\leq\left|\sum_{m=0}^\infty\frac{\left(z^n\right)^m}{m!}\right|\leq\sum_{m=0}^\infty\left|\frac{\left(z^n\right)^m}{m!}\right|\leq\sum_{m=0}^\infty\frac{\left|z\right|^{nm}}{m!}\leq e^{b\left|z\right|^n}
    \]
    which shows that $\rho_{e^{bz^n}}\leq n$. However if we choose the exponent in the
    definition of order to be $b$ we get exactly $e^{bz^n}=e^{Bz^n}$. From this
    we can conclude that the order of $e^{bz^n}$ is exactly $n$.

  \item[(c)] For $e^{e^z}$ suppose that it had order $n$. Then if $z=x+iy$
    \[
      \left|e^{e^z}\right|=\left|e^{e^x\left(\cos x+i\sin y\right)}\right|=e^{e^x\cos x}< Ae^{B\left|z\right|^n}
    \]
    Take the logarithm of both sides to get
    \[
      e^x\cos x < C\left|z\right|^n
    \]
    which is most assuredly not true for all $z\in\bb{C}$. As
    such $e^{e^z}$ does not have finite order.
  \end{enumerate}
\end{proof}

\sk

\begin{problem}[5.6.6]
  Prove Wallis's product formula
  \[
    \frac{\pi}{2}=\prod_{m=1}^\infty\frac{\left(2m\right)^2}{\left(2m-1\right)\left(2m+1\right)}
  \]
  [Hint: Use the product formula for $\sin z$ at $z=\pi/2$.]
\end{problem}

\begin{proof}
  Plugging in $z=\pi/2$ we get
  \[
    1=\frac{\pi}{2}\prod_{n=1}^\infty\left(1-\frac{1}{4n^2}\right)=\frac{\pi}{2}\prod_{n=1}^\infty\frac{\left(2n+1\right)\left(2n-1\right)}{\left(2n\right)^2}
  \]
  Which then dividing by the product gives us
  \[
    \prod_{n=1}^\infty\frac{\left(2n\right)^2}{\left(2n+1\right)\left(2n-1\right)}=\frac{\pi}{2}
  \]
  as desired.
\end{proof}

\sk

\begin{problem}[5.6.8]
  Prove that for every $z$ the product below converges, and
  \[
    \prod_{k=1}^\infty \cos\left(z/2^k\right)=\frac{\sin z}{z}
  \]
  [Hint: Use the fact that $\sin 2z=2\sin z\cos z$.]
\end{problem}

\begin{proof}
  Start with using the suggested identity with $z/2$ to get
  \[
    \sin z=2\sin\left(z/2\right)\cos\left(z/2\right)
  \]
  Then iterate usage of the identity and divide by $z$ to get
  \[
    \frac{\sin z}{z} = \frac{2^N\sin\left(z/2^N\right)}{z}\prod_{k=1}^N\cos\left(z/2^k\right)
  \]
  Use Wallis's formula on the right term to get
  \[
    \frac{2^N\sin\left(z/2^N\right)}{z}=\frac{2^N}{z}\cdot\frac{z}{2^N}\prod_{k=1}^\infty\left(1-\frac{\left(\frac{z}{2^N}\right)^2}{k^2\pi^2}\right)=\prod_{k=1}^\infty\left(1-\frac{\left(\frac{z}{2^N}\right)^2}{k^2\pi^2}\right)
  \]
  which goes to $1$ as $N\rightarrow\infty$. Which in turn gives us that
  \[
    \frac{\sin z}{z}=\prod_{k=1}^\infty\cos\left(z/2^k\right)
  \]
  as desired.
\end{proof}

\sk

\begin{problem}[5.6.10(b)]
  Show that the Hadamard product for $\cos z$ is 
  \[
    \cos\pi z=\prod_{n=0}^\infty 1-\frac{4z^2}{\left(2n+1\right)^2}
  \]
\end{problem}

\begin{proof}
  First note that $\cos z$ has growth order $1$. As such for the Hadamard
  product formula we get
  \begin{align*}
    \cos\pi z&=e^{P\left(z\right)}z^0\prod_{n=-\infty}^\infty E_1\left(2z/\left(2n+1\right)\right)\\
           &=e^{P\left(z\right)}\prod_{n=-\infty}^\infty \left(1-\frac{2z}{2n+1}\right)e^{2z/\left(2n+1\right)}\\
           &=e^{P\left(z\right)}\left(\prod_{n=0}^\infty \left(1-\frac{2z}{2n+1}\right)e^{2z/\left(2n+1\right)}\right)\left(\prod_{n=0}^\infty \left(1+\frac{2z}{2n+1}\right)e^{-2z/\left(2n+1\right)}\right)\\
           &=e^{P\left(z\right)}\prod_{n=0}^\infty\left(1-\frac{4z^2}{\left(2n+1\right)^2}\right)
  \end{align*}
  Now we must show that $P\left(z\right)=0$. Since $P$ is at most degree $1$
  it will be sufficient to show $P$ evaluates to zero at two distinct
  values for $z$. Begin with $z=0$
  \[
    1=e^{P\left(0\right)}\prod_{n=0}^\infty 1
  \]
  Giving that $P\left(0\right)=0$. Then use $z=2$ to get
  \[
    1=e^{P\left(2\right)}\prod_{n=0}^\infty\left(1-\frac{16}{\left(2n+1\right)^2}\right)
  \]
  The partial product for the right term is
  \[
    \prod_{n=0}^N1-\frac{16}{\left(2n+1\right)^2}=\frac{\left(2N+3\right)\left(2N+5\right)}{4N^2-1}
  \]
  Which implies that $\prod_{n=0}^\infty 1-\frac{16}{\left(2n+1\right)^2}=1$ and as
  such $P\left(2\right)=0$ as well. Thus $P\left(z\right)=0$ and we have the formula
  \[
    \cos\pi z=\prod_{n=0}^\infty 1-\frac{4z^2}{\left(2n+1\right)^2}
  \]
\end{proof}

\sk

\begin{problem}[6.3.5]
  Use the fact that $\Gamma\left(s\right)\Gamma\left(1-s\right)=\pi/\sin\pi s$ to prove that
  \[
    |\Gamma\left(1/2+it\right)|=\sqrt{\frac{2\pi}{e^{\pi t}+e^{-\pi t}}}=\sqrt{\pi\sech{\pi t}}
  \]
  whenever $t\in \bb{R}$.
\end{problem}

\begin{proof}
  Use the above identity with $s=1/2+it$ to get
  \begin{align*}
    \Gamma\left(1/2+it\right)\Gamma\left(1/2-it\right) &= \frac{2\pi}{\sin\pi s}\\
                       &= \frac{2\pi i}{e^{i\pi/2-\pi t}-e^{-i\pi/2+\pi t}}\\
                       &= \frac{2\pi i}{i\left(e^{\pi t}-e^{-\pi t}\right)}\\
                       &= \frac{2\pi}{e^{\pi t}-e^{-\pi t}}\\
  \end{align*}
  Then using the fact that $\Gamma\left(\bar{s}\right)=\overline{\Gamma\left(s\right)}$ and
  that $s\bar{s}=|s|^2$ we get
  \[
    |\Gamma\left(1/2+it\right)|^2=\frac{2\pi}{e^{\pi t}-e^{-\pi t}}
  \]
  Which leads us to the desired identity
  \[
    |\Gamma\left(1/2+it\right)|=\sqrt{\frac{2\pi}{e^{\pi t}-e^{-\pi t}}}
  \]
\end{proof}

\sk

\begin{problem}[6.3.7]
  The \textbf{Beta function} is defined for $\Re\left(\alpha\right)>0$ and
  $\Re\left(\beta\right)>0$ by
  \[
    B\left(\alpha,\beta\right)=\int_0^1\left(1-t\right)^{\alpha-1}t^{\beta-1}dt
  \]
  \begin{enumerate}
  \item[(a)] Prove that
    $B\left(\alpha,\beta\right)=\frac{\Gamma\left(\alpha\right)\Gamma\left(\beta\right)}{\Gamma\left(\alpha+\beta\right)}$.
  \item[(b)] Show that
    $B\left(\alpha,\beta\right)=\int_0^\infty\frac{u^{\alpha-1}}{\left(1+u\right)^{\alpha+\beta}}du$.
  \end{enumerate}
  [Hint: For part (a), note that
  \[
    \Gamma\left(\alpha\right)\Gamma\left(\beta\right)=\int_0^\infty\int_0^\infty t^{\alpha-1}s^{\beta-1}e^{-t-s}dtds
  \]
  and make the change of variables $s=ur,t=u\left(1-r\right)$.]
\end{problem}

\begin{proof}
  \begin{enumerate}
  \item[(a)] Begin with $\Gamma\left(\alpha\right)\Gamma\left(\beta\right)$ and make the above suggested
    substitution to get
    \begin{align*}
      \Gamma\left(\alpha\right)\Gamma\left(\beta\right) &= \left(\int_0^\infty e^{-t}t^{\alpha-1}dt\right)\left(\int_0^\infty e^{-s}s^{\beta-1}ds\right)\\
               &=\int_0^\infty\int_0^\infty t^{\alpha-1}s^{\beta-1}e^{-\left(t+s\right)}dtds\\
               &=\int_0^1\int_0^\infty\left(u\left(1-r\right)\right)^{\alpha-1}\left(ur\right)^{\beta-1}e^{-u}\left(-u\right)dudr\\
               &=\int_0^1\int_0^\infty-u^{\alpha+\beta-1}e^{-u}\left(1-r\right)^{\alpha-1}r^{\beta-1}dudr\\
               &=\int_0^1\left(\int_0^\infty-u^{\alpha+\beta-1}e^{-u}du\right)\left(1-r\right)^{\alpha-1}r^{\beta-1}dr\\
               &=\int_0^1\Gamma\left(\alpha+\beta\right)\left(1-r\right)^{\alpha-1}r^{\beta-1}dr\\
               &=\Gamma\left(\alpha+\beta\right)\int_0^1\left(1-r\right)^{\alpha-1}r^{\beta-1}dr\\
               &=\Gamma\left(\alpha+\beta\right)B\left(\alpha,\beta\right)\\
    \end{align*}
    Which when we divide through gives us
    \[
      B\left(\alpha,\beta\right)=\frac{\Gamma\left(\alpha\right)\Gamma\left(\beta\right)}{\Gamma\left(\alpha+\beta\right)}
    \]
  \item[(b)] Begin with the integral side of the problem and make the
    substitution $\frac{1-t}{t}=u$ to get
    \begin{align*}
      \int_0^\infty\frac{u^{\alpha-1}}{\left(1+u\right)^{\alpha+\beta}}du &= \int_0^\infty \left(\frac{u}{1+u}\right)^{\alpha-1}\left(1+u\right)^{-\beta-1}du\\
                                                  &=\int_1^0 \left(1-t\right)^{\alpha-1}\left(t^{-1}\right)^{-\beta-1}\left(-t^{-2}\right)dt\\
                                                  &=\int_0^1 \left(1-t\right)^{\alpha-1}t^{\beta-1}dt\\
                                                  &=B\left(\alpha,\beta\right)
    \end{align*}
  \end{enumerate}
\end{proof}

\sk

\begin{problem}[6.3.10]
  An integral of the form
  \[
    F\left(z\right)=\int_0^\infty f\left(t\right)t^{z-1}dt
  \]
  is called a \textbf{Mellin transform}, and we shall write
  ${\cal M}\left(f\right)\left(z\right)=F\left(z\right)$. For example, the gamma function is the Mellin
  transform of the function $e^{-t}$.
  \begin{enumerate}
  \item[(a)] Prove that
    \[
      {\cal M}\left(\cos\right)\left(z\right)=\int_0^\infty\cos\left(t\right)t^{z-1}dt=\Gamma\left(z\right)\cos\left(\frac{\pi z}{2}\right)
    \]
    for $0<\Re\left(z\right)<1$ and
    \[
      {\cal M}\left(\sin\right)\left(z\right)=\int_0^\infty\sin\left(t\right)t^{z-1}dt=\Gamma\left(z\right)\sin\left(\frac{\pi z}{2}\right)
    \]
    for $0<\Re\left(z\right)<1$.
  \item[(b)] Show that the second of the above is valid in the larger strip
    $-1<\Re\left(z\right)<1$, and that as a consequence, one has
    \[
      \int_0^\infty \frac{\sin x}{x}dx=\frac{\pi}{2}\quad \text{and}\quad \int_0^\infty\frac{\sin x}{x^{3/2}}dx=\sqrt{2\pi}
    \]
  \end{enumerate}
  [Hint: For the first part, consider the integral of the function
  $f\left(w\right)=e^{-w}w^{z-1}$ around the quarter annulus. Use the analytic continuation
  to prove the second part.]
\end{problem}

\begin{proof}
  \begin{enumerate}
  \item[(a)] Let $\gamma_{\epsilon,R}$ be the quarter annulus in the upper right quarter
    plane with inner radius $\epsilon$ and outer radius $R$. Since there are no
    singularities within for $f\left(w\right)$, the value of the integral is zero. Thus we have
    \[
      \int_{\gamma_{\epsilon,R}}e^{-w}w^{z-1}dw = 0 = \int_\epsilon^R e^{-t}t^{z-1}dt + \int_\epsilon^R e^{-it}t^{z-1}idt+\int_{C_\epsilon}e^{-w}w^{z-1}dw+\int_{C_R}e^{-w}w^{z-1}dw
    \]
    The leftmost term in the sum is $\Gamma\left(z\right)$, the two integrals about $C_\epsilon$ and $C_R$
    will both approach zero as $\epsilon\rightarrow 0$ and $R\rightarrow \infty$. This gives us
    \begin{align*}
      0 &= \Gamma\left(z\right)+i^z\int_0^\infty e^{-it}t^{z-1}dt=\Gamma\left(z\right)+i^z\left(\int_0^\infty\cos\left(t\right)t^{z-1}dt-i\int_0^\infty\sin\left(t\right)t^{z-1}dt\right)\\
        &=\Gamma\left(z\right)+i^z\left({\cal M}\left(\cos\right)\left(z\right)-i{\cal M}\left(\sin\right)\left(z\right)\right)
    \end{align*}

    Rearrange to get
    \[
      {\cal M}\left(\cos\right)\left(z\right)-i{\cal M}\left(\sin\right)\left(z\right)= \Gamma\left(z\right)i^{-z}=\Gamma\left(z\right)\cos\left(\pi z/2\right)-i\Gamma\left(z\right)\sin\left(\pi z/2\right)
    \]
    If $z$ is real then by taking the real and imaginary parts of the above
    equality we get
    \[
      {\cal M}\left(\cos\right)\left(z\right)=\int_0^\infty\cos\left(t\right)t^{z-1}dt=\Gamma\left(z\right)\cos\left(\frac{\pi z}{2}\right)
    \]
    and
    \[
      {\cal M}\left(\sin\right)\left(z\right)=\int_0^\infty\sin\left(t\right)t^{z-1}dt=\Gamma\left(z\right)\sin\left(\frac{\pi z}{2}\right)
    \]

    
    For ${\cal M}\left(\cos\right)\left(z\right)$ the original integral exists in the strip defined
    by $0<\Re\left(z\right)<1$. Since the integral agrees with $\Gamma\left(z\right)\cos\left(\pi z/2\right)$ on
    interval $\left(0,1\right)$ and $\Gamma\left(z\right)\cos\left(\pi z/2\right)$ is analytic in said strip it
    must be that ${\cal M}\left(\cos\right)\left(z\right)$ continues analytically to the whole
    strip with values $\Gamma\left(z\right)\cos\left(\pi z/2\right)$. The same reasoning applies
    to ${\cal M}\left(\sin\right)\left(z\right)$.
  \item[(b)] Using that $\Gamma\left(z\right)=\left(z-1\right)\Gamma\left(z-1\right)$ and that $\cos\left(z\right)=\sin\left(z+\pi/2\right)$
    we can continue ${\cal M}\left(\sin\right)\left(z\right)$ to the larger strip by
    \[
      {\cal M}\left(\sin\right)\left(z-1\right)=\Gamma\left(z-1\right)\sin\left(\pi/2\left(z-1\right)\right)=\frac{\Gamma\left(z\right)\cos\left(\pi z/2\right)}{z-1}
    \]
    Now if we plug-in $-1/2$ in for $z$ we get
    \[
      {\cal M}\left(\sin\right)\left(-1/2\right)=\int_0^\infty \frac{\sin\left(t\right)}{t^{3/2}}dt = \frac{\Gamma\left(1/2\right)\cos\left(-\pi/4\right)}{-1/2}=\frac{2\sqrt{\pi}}{\sqrt{2}}=\sqrt{2\pi}
    \]
    Using $z=0$ we have
    \begin{align*}
      {\cal M}\left(\sin\right)\left(0\right)=\int_0^\infty\frac{\sin\left(t\right)}{t}dt &= \lim_{z\rightarrow 0}\Gamma\left(z\right)\sin\left(\pi z/2\right)\\
                                                 &= \lim_{z\rightarrow 0}\Gamma\left(z+1\right)\frac{\sin\left(\pi z/2\right)}{z}\\
                                                 &=\left(\lim_{z\rightarrow 0}\Gamma\left(z+1\right)\right)\left(\lim_{z\rightarrow 0}\frac{\sin\left(\pi z/2\right)}{z}\right)\\
                                                 &=1\lim_{z\rightarrow 0}\frac{\sin\left(\pi z/2\right)}{z}\\
                                                 &=\lim_{z\rightarrow 0}\frac{\pi/2\cos\left(\pi z/2\right)}{1}\\
                                                 &=\pi/2
    \end{align*}
  \end{enumerate}
\end{proof}

\sk

\begin{problem}[6.3.15]
  Prove that for $\Re\left(s\right)>1$,
  \[
    \zeta\left(s\right)=\frac{1}{\Gamma\left(s\right)}\int_0^\infty\frac{x^{s-1}}{e^x-1}dx
  \]
  [Hint: Write $1/\left(e^x-1\right)=\sum_{n=1}^\infty e^{-nx}$.]
\end{problem}

\begin{proof}
  Start with the right side of the equation and use the suggested series
  with a substitution $t=nx$ to get
  \begin{align*}
    \frac{1}{\Gamma\left(s\right)}\int_0^\infty \frac{x^{s-1}}{e^x-1}dx &= \frac{1}{\Gamma\left(s\right)}\int_0^\infty\sum_{n=1}^\infty x^{s-1}e^{-nx}dx\\
                                                &=\frac{1}{\Gamma\left(s\right)}\sum_{n=1}^\infty \int_0^\infty e^{-nx}dx\\
                                                &=\frac{1}{\Gamma\left(s\right)}\sum_{n=1}^\infty \int_0^\infty n^{-s}t^{s-1}e^{-t}dt\\
                                                &=\frac{1}{\Gamma\left(s\right)}\sum_{n=1}^\infty n^{-s}\int_0^\infty t^{s-1}e^{-t}dt\\
                                                &=\frac{1}{\Gamma\left(s\right)}\sum_{n=1}^\infty n^{-s}\Gamma\left(s\right)\\
                                                &=\sum_{n=1}^\infty n^{-s}\\
                                                &= \zeta\left(s\right)
  \end{align*}
\end{proof}

%%%%%%%%%%%%%%%%%%%%%%%%%%%%%%%%%%%%%%%%%%%%%%%%%%%%%%%%%%%%%%%%%%%%%%%%%%%%%
\end{document}
