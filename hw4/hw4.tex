\documentclass[10pt]{article}
\usepackage[utf8]{inputenc}
\usepackage{amscd}
\usepackage{amsmath}
\usepackage{amssymb}
\usepackage{amsthm}
\usepackage{listings}
\usepackage{enumerate}
\usepackage[all,cmtip]{xy}

\textwidth=15cm \textheight=22cm \topmargin=0.5cm \oddsidemargin=0.5cm \evensidemargin=0.5cm

\newcommand{\sk}{\vskip 10mm}
\newcommand{\bb}[1]{\mathbb{#1}}
\newcommand{\ra}{\rightarrow}
\newcommand{\conj}[1]{\overline{#1}}
\newcommand{\wt}[1]{\widetilde{#1}}
\DeclareMathOperator{\cis}{cis}
\DeclareMathOperator{\res}{Res}
\DeclareMathOperator{\sech}{sech}


\theoremstyle{plain}
\newtheorem{problem}{Problem}
\newtheorem{lemma}{Lemma}[problem]

\theoremstyle{remark}
\newtheorem{tpart}{}[problem]
\newtheorem*{ppart}{}

\begin{document}

\begin{problem}[5.6.2]
  Find the order of growth of the following entire functions:
  \begin{enumerate}
  \item[(a)] $p(z)$ where $p$ is a polynomial.
  \item[(b)] $e^{bz^n}$ for $b\neq 0$.
  \item[(c)] $e^{e^z}$.
  \end{enumerate}
\end{problem}

\begin{proof}
  \begin{enumerate}
  \item[(a)] Let $n=\deg p(z)$ and $\rho>0$. Choose $C>|a_0|$ such that $|p(z)|\leq C|z|^n$
    and $m$ so that $\rho m>n$. Then
    \[
      m!Ce^{|z|^\rho}=m!C\sum_{k=0}^\infty\frac{|z|^{\rho k}}{k!}\geq C|z|^{\rho m}\geq C|z|^n\geq|p(z)|
    \]
    Since this holds for any $\rho>0$ we have that $\rho_{p(z)}=0$.
  \item[(b)] Using the Taylor expansion we get
    \[
      |e^{bz^n}|\leq|\sum_{m=0}^\infty\frac{(z^n)^m}{m!}|\leq\sum_{m=0}^\infty|\frac{(z^n)^m}{m!}|\leq\sum_{m=0}^\infty\frac{|z|^{nm}}{m!}\leq e^{b|z|^n}
    \]
    which shows that $\rho_{e^{bz^n}}\leq n$. However if we choose the exponent in the
    definition of order to be $b$ we get exactly $e^{bz^n}=e^{Bz^n}$. From this
    we can conclude that the order of $e^{bz^n}$ is exactly $n$.

  \item[(c)]
  \end{enumerate}
\end{proof}

\sk

\begin{problem}[5.6.6]
  Prove Wallis's product formula
  \[
    \frac{\pi}{2}=\prod_{m=1}^\infty\frac{(2m)^2}{(2m-1)(2m+1)}
  \]
  [Hint: Use the product formula for $\sin z$ at $z=\pi/2$.]
\end{problem}

\begin{proof}
  
\end{proof}

\sk

\begin{problem}[5.6.8]
  Prove that for every $z$ the product below converges, and
  \[
    \prod_{k=1}^\infty \cos(z/2^k)=\frac{\sin z}{z}
  \]
  [Hint: Use the fact that $\sin 2z=2\sin z\cos z$.]
\end{problem}

\begin{proof}
  
\end{proof}

\sk

\begin{problem}[5.6.10(b)]
  Show that the Hadamard product for $\cos z$ is 
  \[
    \cos\pi z=\prod_{n=0}^\infty 1-\frac{4z^2}{(2n+1)^2}
  \]
\end{problem}

\begin{proof}
  
\end{proof}

\sk

\begin{problem}[6.3.5]
  Use the fact that $\Gamma(s)\Gamma(1-s)=\pi/\sin\pi s$ to prove that
  \[
    |\Gamma(1/2+it)|=\sqrt{\frac{2\pi}{e^{\pi t}+e^{-\pi t}}}=\sqrt{\pi\sech{\pi t}}
  \]
  whenever $t\in \bb{R}$.
\end{problem}

\begin{proof}
  
\end{proof}

\sk

\begin{problem}[6.3.7]
  The \textbf{Beta function} is defined for $\Re(\alpha)>0$ and
  $\Re(\beta)>0$ by
  \[
    B(\alpha,\beta)=\int_0^1(1-t)^{\alpha-1}t^{\beta-1}dt
  \]
  \begin{enumerate}
  \item[(a)] Prove that
    $B(\alpha,\beta)=\frac{\Gamma(\alpha)\Gamma(\beta)}{\Gamma(\alpha+\beta)}$.
  \item[(b)] Show that
    $B(\alpha,\beta)=\int_0^\infty\frac{u^{\alpha-1}}{(1+u)^{\alpha+\beta}}du$.
  \end{enumerate}
  [Hint: For part (a), note that
  \[
    \Gamma(\alpha)\Gamma(\beta)=\int_0^\infty\int_0^\infty t^{\alpha-1}s^{\beta-1}e^{-t-s}dtds
  \]
  and make the change of variables $s=ur,t=u(1-r)$.]
\end{problem}

\begin{proof}
  
\end{proof}

\sk

\begin{problem}[6.3.10]
  An integral of the form
  \[
    F(z)=\int_0^\infty f(t)t^{z-1}dt
  \]
  is called a \textbf{Mellin transform}, and we shall write
  ${\cal M}(f)(z)=F(z)$. For example, the gamma function is the Mellin
  transform of the function $e^{-t}$.
  \begin{enumerate}
  \item[(a)] Prove that
    \[
      {\cal M}(\cos)(z)=\int_0^\infty\cos(t)t^{z-1}dt=\Gamma(z)\cos(\frac{\pi z}{2})
    \]
    for $0<\Re(z)<1$ and
    \[
      {\cal M}(\sin)(z)=\int_0^\infty\sin(t)t^{z-1}dt=\Gamma(z)\sin(\frac{\pi z}{2})
    \]
    for $0<\Re(z)<1$.
  \item[(b)] Show that the second of the above is valid in the larger strip
    $-1<\Re(z)<1$, and that as a consequence, one has
    \[
      \int_0^\infty \frac{\sin x}{x}dx=\frac{\pi}{2}\quad \text{and}\quad \int_0^\infty\frac{\sin x}{x^{3/2}}dx=\sqrt{2\pi}
    \]
  \end{enumerate}
  [Hint: For the first part, consider the integral of the function
  $f(w)=e^{-w}w^{z-1}$ around the quarter annulus. Use the analytic continuation
  to prove the second part.]
\end{problem}

\begin{proof}
  
\end{proof}

\sk

\begin{problem}[6.3.15]
  Prove that for $\Re(s)>1$,
  \[
    \zeta(s)=\frac{1}{\Gamma(s)}\int_0^\infty\frac{x^{s-1}}{e^x-1}dx
  \]
  [Hint: Write $1/(e^x-1)=\sum_{n=1}^\infty e^{-nx}$.]
\end{problem}

\begin{proof}
  
\end{proof}

%%%%%%%%%%%%%%%%%%%%%%%%%%%%%%%%%%%%%%%%%%%%%%%%%%%%%%%%%%%%%%%%%%%%%%%%%%%%%
\end{document}
