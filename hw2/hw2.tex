\documentclass[10pt]{article}
\usepackage[utf8]{inputenc}
\usepackage{amscd}
\usepackage{amsmath}
\usepackage{amssymb}
\usepackage{amsthm}
\usepackage{listings}
\usepackage{enumerate}
\usepackage[all,cmtip]{xy}

\textwidth=15cm \textheight=22cm \topmargin=0.5cm \oddsidemargin=0.5cm \evensidemargin=0.5cm

\newcommand{\sk}{\vskip 10mm}
\newcommand{\bb}[1]{\mathbb{#1}}
\newcommand{\ra}{\rightarrow}
\newcommand{\conj}[1]{\overline{#1}}
\DeclareMathOperator{\cis}{cis}
\DeclareMathOperator{\Res}{Res}
\DeclareMathOperator{\Arg}{Arg}

\theoremstyle{remark}
\newtheorem{problem}{Problem}
\newtheorem{lemma}{Lemma}[problem]

\theoremstyle{remark}
\newtheorem{tpart}{}[problem]
\newtheorem*{ppart}{}

\begin{document}

\begin{problem}[2.6.14]
  Suppose that $f$ is holomorhic in an open set containing the closed unit disc,
  expcept for a pole at $z_0$ on the unit circle. Show that if
  \[
    \sum_{n=0}^\infty a_n z^n
  \]
  denotes the power series expansion of $f$ in the open unit disc, then
  \[
    \lim_{n\rightarrow\infty}\frac{a_n}{a_{n+1}}=z_0.
  \]
  
  Only need to do case where degree of pole is greater than 1.
\end{problem}

\begin{proof}
  We proceed by induction over the degree of the pole. The case of the simple
  pole was handled in class. Suppose that the above property holds for poles of
  degree $k$. Then if $z_0$ is a pole of degree $k+1$ we can construct a function
  \[
    g(z)=(z-z_0)f(z)
  \]
  This function has a pole of degree $k$. By induction for the power series
  $g(z)=\sum_0^\infty b_nz^n$ we have that $\lim_{n\rightarrow\infty}\frac{b_n}{b_{n+1}}=z_0$. Since
  power series are unique. We can write the coefficients $b_n$ in terms of $a_n$.
  \[
    f(z)(z-z_0)=\sum_0^\infty(z-z_0)a_nz^n
  \]
  This implies that $b_n=a_{n-1}-a_nz_0$ where $a_{n-1}=0$. Now we rewrite the limit
  as
  \[
    \lim_{n\rightarrow\infty}\frac{b_n}{b_{n+1}}=\lim_{n\rightarrow\infty}\frac{a_{n-1}-a_nz_0}{a_n-a_{n+1}z_0}=z_0
  \]
  and use it to show that $\lim_{n\rightarrow\infty}\frac{a_n}{a_{n+1}}=z_0$.

  To do this first we factor out $a_n$ from the top and $a_{n+1}$ from the bottom. This
  gives us
  \[
    \lim_{n\rightarrow\infty}\frac{a_{n-1}-a_nz_0}{a_n-a_{n+1}z_0}=\lim_{n\rightarrow\infty}\frac{a_n}{a_{n+1}}\cdot\frac{\frac{a_{n-1}}{a_n}-z_0}{\frac{a_n}{a_{n+1}}-z_0}
  \]
  We can split the limit since they both exist. The right limit converges to one giving us
  \[
    \lim_{n\rightarrow\infty}\frac{a_n}{a_{n+1}}\cdot\frac{\frac{a_{n-1}}{a_n}-z_0}{\frac{a_n}{a_{n+1}}-z_0}=\lim_{n\rightarrow\infty} \frac{a_n}{a_{n+1}}\cdot\lim_{n\rightarrow\infty}\frac{\frac{a_{n-1}}{a_n}-z_0}{\frac{a_n}{a_{n+1}}-z_0} = \lim_{n\rightarrow\infty}\frac{a_n}{a_{n+1}}=z_0
  \]
  which gives us the desired limit.
\end{proof}

\sk

\begin{problem}[3.8.2]
  Evaluate the integral
  \[
    \int_{-\infty}^\infty \frac{dx}{1+x^4}dx
  \]

  Where are the poles of $1/(1+z^4)$?
\end{problem}

The poles of the function are at $\pm e^{i\pi/4},\pm e^{3i\pi/4}$.

We evaluate this integral by integrating over the semicircle in the
upper half plane of radius $R$ which we will call $\gamma_R$. Then we have
\[
  \int_{\gamma_R}\frac{1}{1+z^4} dz = \int_{-\infty}^\infty \frac{1}{1+x^4}dx+\int_{C_R}\frac{1}{1+z^4}dz= 2\cap i\Res_{z=e^{i\pi/4},e^{3i\pi/4}}\frac{1}{1+z^4}
\]

However since $\frac{1}{1+z^4}\leq\frac{1}{R^4-1}$ on $C_R$ we have
\[
  \left|\int_{C_R}\frac{1}{1+z^4}dz\right|\leq\frac{\pi R}{R^4-1}
\]
which approaches $0$ as $R$ approaches infinity. This gives us that
\[
  \int_{-\infty}^\infty \frac{1}{1+x^4}dx+\int_{C_R}\frac{1}{1+z^4}dz= 2\pi i\Res_{z=e^{i\pi/4},e^{3i\pi/4}}\frac{1}{1+z^4}
\]

To calculate the residues we use the formula from the book. In this case the
formula boils down to plugging in the point to $\frac{1}{1+z^4}$ where removing the factor
associated with the singularity. This gives us
\[
  \Res_{z=e^{i\pi/4}}\frac{1}{1+z^4}=\frac{1}{(e^{2\pi i/4}+e^{\pi i/2})(2e^{i\pi/4})}=\frac{1}{4e^{3\pi i/4}}
\]
and
\[
  \Res_{z=e^{3i\pi/4}}\frac{1}{1+z^4}=\frac{1}{(e^{6\pi i/4}+e^{3\pi i/2})(2e^{3i\pi/4})}=\frac{1}{4e^{9\pi i/4}}
\]
Add them together and we get $\frac{-i\sqrt{2}}{4}$. Multiply it by $2\pi i$ and
we get $\frac{\pi}{2}$. Which gives us that
\[
  \int_{-\infty}^\infty \frac{dx}{1+x^4} = \frac{\pi}{\sqrt{2}}
\]

\sk

\begin{problem}[3.8.4]
  Show that
  \[
    \int_{-\infty}^\infty\frac{x\sin x}{x^2+a^2}dx=\pi e^{-a},\quad \text{for all $a>0$}
  \]
\end{problem}

\begin{proof}
  First note that $\Im(\frac{ze^{iz}}{z^2+a^2})=\frac{z\sin z}{z^2+a^2}$. We proceed
  by integrating the semicircle in the upper half plane, $\gamma_R$, which we can
  expand as
  \[
    \int_{\gamma_R}\frac{ze^{iz}}{z^2+a^2}dz=\int_{C_R}\frac{ze^{iz}}{z^2+a^2}dz+\int_{-R}^R\frac{ze^{iz}}{z^2+a^2} dz = 2\pi i\Res_{z=ia}\frac{ze^{iz}}{z^2+a^2}
  \]

  First we will show that $\left|\int_{C_R}\frac{ze^{iz}}{z^2+a^2}dz\right|\rightarrow 0$ as $R\rightarrow\infty$. Start by
  making the substitution $z=Re^{i\theta}$. Then we get
  \begin{align*}
    \left|\int_{C_R}\frac{ze^{iz}}{z^2+a^2}dz\right| &= \left|\int_0^\pi\frac{Re^{i\theta}e^{iRe^{i\theta}}}{R^2e^{2i\theta}+a^2}Rie^{i\theta}d\theta\right|\\
                                                  &\leq \left|\int_0^\pi\frac{iR^2e^{2i\theta}e^{iRe^{i\theta}}}{R^2-a^2}d\theta\right|\\
                                                  &\leq\left|\int_0^\pi\frac{R^2e^{iRe^{i\theta}}}{R^2-a^2}d\theta\right|\\
                                                  &\leq\frac{R^2}{R^2-a^2}\left|\int_0^\pi e^{iR\cos\theta-R\sin\theta}d\theta\right|\\
                                                  &\le\frac{R^2}{R^2-a^2}\left|\int_0^\pi e^{-R\sin\theta}d\theta\right|
  \end{align*}
  At this point we use two inequalities related to $\sin\theta$. The first
  is that $\sin\theta\geq 2/\pi\theta$ if $0\leq\theta\leq\pi/2$ and the other is that
  $\sin\theta\geq 1-2/\pi\theta$ when $\pi/2\leq\theta\leq\pi$. This gives us \textbf{Lucas you did things horribly wrong}
  \begin{align*}
    \frac{R^2}{R^2-a^2}\left|\int_0^\pi e^{-R\sin\theta}d\theta\right| &\leq\frac{R^2}{R^2-a^2}\left(\int_0^{\pi/2}e^{-R\pi\theta/2}d\theta+\int_{\pi/2}^{\pi}e^{-R+2/\pi R\theta}d\theta\right)\\
                                                               &=\frac{R^2}{R^2-a^2}\left(\frac{-2e^{-R\pi/2\theta}}{\pi R}\right|_0^{\pi/2}+\frac{R^2}{R^2-a^2}\left(\frac{\pi e^{-R+2/\pi R\theta}}{2R}\right|_{\pi/2}^\pi
  \end{align*}% ))
  At this point it is clear that the above approaches zero as $R$ approaches infinity.
  Thus
  \[
    \int_{-R}^R \frac{ze^{iz}}{z^2+a^2}dz = 2\pi i\Res_{z=ia}\frac{ze^{iz}}{z^2+a^2}
  \]

  To calculate the residue we compute
  \begin{align*}
    \lim_{z\rightarrow ia}(z-ia)\frac{ze^{iz}}{z^2+a^2}&=\frac{iae^{-a}}{2ia}\\
                                             &=\frac{e^{-a}}{2}\\
  \end{align*}

  When we put it all together we get
  \begin{align*}
    \int_{-\infty}^\infty\frac{x\sin x}{x^2+a^2}dx&=\Im\left(\int_{-\infty}^\infty\frac{ze^{iz}}{z^2+a^2} dz\right)\\
                                               &=\Im(2\pi i\cdot \frac{e^{-a}}{2})\\
                                               &=\pi e^{-a}
  \end{align*}
\end{proof}

\sk

\begin{problem}[3.8.8]
  Prove that
  \[
    \int_0^{2\pi}\frac{d\theta}{a+b\cos\theta}=\frac{2\pi}{\sqrt{a^2-b^2}}
  \]
  if $a>|b|$ and $a,b\in\bb{R}$.
\end{problem}

\begin{proof}
  We start by rewriting $\cos\theta$ in terms of $e^{i\theta}$ which gives us
  \begin{align*}
    \int_0^{2\pi}\frac{d\theta}{a+b\cos\theta} &= \int_0^{2\pi}\frac{d\theta}{a+b/2(e^{i\theta}+e^{-i\theta})}\\
                                            &= 2\int_0^{2\pi}\frac{e^{i\theta}}{2ae^{i\theta}+be^{2i\theta}+b}d\theta\\
  \end{align*}
  Then substitute $z=e^{i\theta}$ to get
  \begin{align*}
    2\int_0^{2\pi}\frac{e^{i\theta}}{2ae^{i\theta}+be^{2i\theta}+b}d\theta&= \frac{2}{i}\int_{C_1}\frac{1}{bz^2+2az+b}dz\\
                                                                     &= 4\pi\Res\frac{1}{bz^2+2az+b}
  \end{align*}

  First we find the poles by factoring the bottom
  \[
    \frac{1}{bz^2+2az+b}=\frac{1}{b(z-(-a/b+\sqrt{a^2/b^2-1}))(z-(-a/b-\sqrt{a^2/b^2-1}))}
  \]
  The pole that occurs within the circle of radius $1$ is $-a/b+\sqrt{a^2/b^2-1}$. We
  calculate the residue as
  \begin{align*}
    \Res_{z=-a/b+\sqrt{a^2/b^2-1}} \frac{1}{bz^2+2az+b} &= \lim_{z\rightarrow -a/b+\sqrt{a^2/b^2-1}}\frac{(z-(-a/b+\sqrt{a^2/b^2-1}))}{b(z-(-a/b+\sqrt{a^2/b^2-1}))(z-(-a/b-\sqrt{a^2/b^2-1}))}\\
                                                     &=\frac{1}{b(-a/b+\sqrt{a^2/b^2-1}-(-a/b-\sqrt{a^2/b^2-1}))}\\
                                                     &=\frac{1}{2b\sqrt{a^2/b^2-1}}\\
                                                     &=\frac{1}{2\sqrt{a^2-b^2}}
  \end{align*}

  Which when we plug back in we get
  \[
    \int_0^{2\pi}\frac{d\theta}{a+b\cos\theta}=\frac{4\pi}{2\sqrt{a^2-b^2}}=\frac{2\pi}{\sqrt{a^2-b^2}}
  \]
\end{proof}

\sk

\begin{problem}[3.8.9]
  Show that
  \[
    \int_0^1\log(\sin\pi x)dx = -\log 2
  \]

  Hint: Use contour that goes down to 0 and up from 1.
\end{problem}

\begin{proof}
  First note that
  \begin{align*}
    \log(1-e^{2\pi iz}) &=\log(e^{\pi iz}(-2i)(e^{\pi iz}+e^{-\pi iz})/2i)\\
                      &=\pi iz + \log(-2i)+\log(\sin\pi z)\\
                      &= \pi iz + \log(2)-i\pi/2+\log(\sin\pi z)\\
  \end{align*}
  If we integrate this with respect to $z$ from $0$ to $1$ we get
  \[
    \int_0^1\log(1-e^{2\pi iz})dz=\pi i/2+\log(2)-i\pi/2+\int_0^1\log(\sin\pi z) dz = \log(2)+\int_0^1\log(\sin\pi z) dz
  \]
  This will give us the desired equality if we show that $\int_0^1\log(1-e^{2\pi iz})dz=0$. To do this
  we integrate over the rectangle of width $1$, height $R$, with two quarter circles of radius $\epsilon$
  on the corners at $0$ and $1$ to avoid the branch points. Refer to this curve as $\gamma_{\epsilon,R}$. The
  entire integral will be zero as $\log(1-e^{2\cap iz})$ is holomorphic on the curve and its interior.
  Then we split the integral into six pieces
  \begin{align*}
    0 = \int_{\gamma_{\epsilon,R}}\log(1-e^{2\pi iz})dz &= \int_\epsilon^{1-\epsilon}\log(1-e^{2\pi ix}) dx & z=x\\
                                                     &+ \int_{\pi}^{\pi/2}\log(1-e^{2\pi i(1+\epsilon e^{i\theta})})i\epsilon e^{i\theta}d\theta & z=1+\epsilon e^{i\theta}\\
                                                     &+ i\int_\epsilon^R\log(1-e^{2\pi i(1+it)})dt & z=1+it\\
                                                     &+ \int_1^0\log(1-e^{2\pi i(t+iR)})dt & z=t+iR\\
                                                     &+ i\int_R^\epsilon\log(1-e^{2\pi i(it)}) dt & z=it\\
                                                     &+ \int_{\pi/2}^0\log(1-e^{2\pi i \epsilon e^{i\theta}}) d\theta & z=\epsilon e^{i\theta}
  \end{align*}
  First note that for the two vertical portions of $\gamma_{\epsilon,R}$, the third and fifth, that
  \[
    \log(1-e^{2\pi i(1+it)}) = \log(1-e^{2\pi i(it)})
  \]
  as $e^{2\pi i }=1$. Since the only difference then is that the limits of integration are swapped
  these two cancel eachother out. What is left to show is that
  \[
    \left|\int_1^0\log(1-e^{2\pi i(t+iR)})dt\right|,\quad\left|\int_{\pi}^{\pi/2}\log(1-e^{2\pi i\epsilon e^{i\theta}})i\epsilon e^{i\theta}d\theta\right|,\quad \left|\int_{\pi/2}^0\log(1-e^{2\pi i \epsilon e^{i\theta}}) d\theta\right|
  \]
  all approach zero as $\epsilon\rightarrow 0 $ and $R\rightarrow\infty$.

  Starting with the first we have
  \begin{align*}
    \left|\int_0^1\log(1-e^{2\pi i(t+iR)})dt\right| &\leq \int_0^1\left|\log(1-e^{-2\pi R}e^{2\pi i t})\right|dt\\
                                              &\leq\int_0^1\log|1-e^{-2\pi R}e^{2\pi i t}|+|i\Arg(1-e^{-2\pi R}e^{2\pi it})|dt\\
                                              & \leq\int_0^1\log(|1|+|e^{-2\pi R}e^{2\pi i t}|)+|i\Arg(1-e^{-2\pi R}e^{2\pi it})|dt\\
                                              &\leq\int_0^1\log(1+e^{-2\pi R})+|i\Arg(1-e^{-2\pi R}e^{2\pi it})|dt\\
                                              &\leq\int_0^1\log(1+e^{-2\pi R})+|\Arg(1-e^{-2\pi R}e^{2\pi it})|dt\\
                                              &\leq\int_0^1\log(1+e^{-2\pi R})+|\arctan(\frac{\Im(1-e^{-2\pi R}e^{2\pi it})}{\Re(1-e^{-2\pi R}e^{2\pi it})})|dt\\
                                              &\leq\int_0^1\log(1+e^{-2\pi R})+|\arctan(\frac{e^{-2\pi R}\sin(2\pi t)}{1-e^{2\pi R}\cos(2\pi t)})|dt\\
                                              &\leq\int_0^1\log(1+e^{-2\pi R})+|\frac{e^{-2\pi R}\sin(2\pi t)}{1-e^{2\pi R}\cos(2\pi t)}|dt\\
                                              &\leq\int_0^1\log(1+e^{-2\pi R})+|\frac{e^{-2\pi R}}{1-e^{2\pi R}}|dt\\
                                              &= \log(1+e^{-2\pi R})+|\frac{e^{-2\pi R}}{1-e^{2\pi R}}|
  \end{align*}
  which goes to zero as $R\rightarrow \infty$.

  \textbf{Check logarithm inequalities. Might not work for $\epsilon$}.
  Next we show that $|\int_{\pi/2}^0\log(1-e^{2\pi i \epsilon e^{i\theta}}) d\theta|\rightarrow 0$ as $\epsilon\rightarrow 0$.
  \begin{align*}
    |\int_{\pi/2}^0\log(1-e^{2\pi i \epsilon e^{i\theta}}) \epsilon ie^i\theta d\theta| &\leq \int_0^{\pi/2}|\log(1-e^{2\pi i \epsilon e^{i\theta}})\epsilon ie^{i\theta}|d\theta\\
                                                       &\leq\epsilon\int_0^{\pi/2}|\log(1-e^{2\pi i \epsilon e^{i\theta}})|d\theta\\
                                           &\leq \epsilon\int_0^{\pi/2}|\log|1-e^{2\pi i \epsilon e^{i\theta}}|+i\Arg(1-e^{2\pi i \epsilon e^{i\theta}})|d\theta\\
                                                       &\leq \epsilon\int_0^{\pi/2}\log(1+|e^{2\pi i\epsilon e^{i\theta}})+|\Arg(1-e^{2\pi i \epsilon e^{i\theta}})|d\theta\\
                                                       &\leq\epsilon\int_0^{\pi/2}\log(1+|e^{2\pi i\epsilon e^{i\theta}}|)+\pi d\theta\\
                                                       &\leq\epsilon\int_0^{\pi/2}\log(1+e^{-2\pi \epsilon\sin\theta|})+\pi d\theta\\
                                                       &\leq\epsilon\int_0^{\pi/2}\log(1+e^{2\pi \epsilon})+\pi d\theta\\
                                                       &= \epsilon\cdot\frac{\pi}{2}\log(1+e^{2\pi\epsilon})
  \end{align*}
  Which at this point clearly goes to zero as $\epsilon\rightarrow 0$. The last quarter-circle integral is identical
  once we note that
  \[
    \int_{\pi}^{\pi/2}\log(1-e^{2\pi i(1+\epsilon e^{i\theta})})i\epsilon e^{i\theta}d\theta =  \int_{\pi}^{\pi/2}\log(1-e^{2\pi i(\epsilon e^{i\theta})})i\epsilon e^{i\theta}d\theta
  \]
  We can then bound this integral the same as the prior. Since the arc length of the quarter-circle
  is the same we can safely conclude that this integral also approaches zero as $\epsilon\rightarrow 0$.

  Therefore, since every portion of $\int_{\gamma_{\epsilon,R}}\log(1-e^{2\pi i z})$ is zero aside from
  $\int_{-\epsilon}^\epsilon\log(\sin(\pi x))dx$, we can conclude that
  \[
    \int_0^1\log(\sin\pi x)dx = -\log 2
  \]
\end{proof}

\sk

\begin{problem}[3.8.10]
  Show that if $a>0$, then
  \[
    \int_0^\infty\frac{\log x}{x^2+a^2}dx=\frac{\pi}{2a}\log a
  \]

  Hint: Integrate over upper half annulus with inner radius $\epsilon$ and outer
  radius $R$.
\end{problem}

\begin{proof}
  Let $\gamma_{\epsilon,R}$ denote the curve around half annulus in the upper half plane with
  inner radius $\epsilon$ and outer radius $R$. The function $\frac{\log x}{x^2+a^2}$ has
  a pole at $ia$. Then the integral over $\gamma_{\epsilon,R}$ is
  \begin{align*}
    \int_{\gamma_{\epsilon,R}}\frac{\log z}{z^2+a^2} dz&= 2\pi i\Res_{z=ia}\frac{\log z}{z^2+a^2}\\
                                                    &= \int_{C_R}\frac{\log z}{z^2+a^2}dz \\
                                                    &+ \int_{C_\epsilon}\frac{\log z}{z^2+a^2}dz \\
                                                    &+ \int_\epsilon^R \frac{\log x}{x^2+a^2}dx\\
                                                    &-\int_{R}^{\epsilon}\frac{\log -x}{x^2+a^2}dx
  \end{align*}
  First we show that the integral on $C_R$ approaches $0$ as $R\rightarrow\infty$.
  \begin{align*}
    |\int_{C_R}\frac{\log z}{z^2+a^2}dz| &\leq \pi R\cdot \frac{R+i\pi}{R^2-a^2}\\
  \end{align*}
  This approaches zero as $R$ approaches infinity.

  Next we show that the integral along $C_\epsilon$ approaches $0$ as $\epsilon\rightarrow 0$.
  \textbf{Finish this}
  \begin{align*}
    |\int_{C_\epsilon}\frac{\log z}{z^2+a^2}dz & \leq \\
  \end{align*}
  
  The residue calculation is
  \[
    2\pi i\Res_{z=ia}\frac{\log z}{z^2+a^2}= 2\pi i\frac{\log ia}{2ia} = \pi\log a+\frac{i\pi^2}{2}
  \]

  Note that for the last integral, since we are on the principle branch we have
  \[
    -\int_{R}^{\epsilon}\frac{\log -x}{x^2+a^2}dx =\int_\epsilon^R\frac{\log x}{x^2+a^2}dx+\int_\epsilon^R\frac{i\pi}{x^2+a^2}dx
  \]
  
  Which gives us that
  \[
    2\int_0^\infty\frac{\log x}{x^2+a^2}dx+i\pi\int_0^\infty\frac{1}{x^2+a^2}dx = \pi\log a+\frac{i\pi^2}{2}
  \]

  Since we know that $i\pi\int_0^\infty\frac{1}{x^2+a^2}dx = \frac{i\pi^2}{2}$ we get
  \[
    \int_0^\infty\frac{\log x}{x^2+a^2}dx=\frac{\pi\log a}{2}
  \]
  as desired.
\end{proof}

\sk

\begin{problem}[3.8.13]
  Suppose $f$ is holomorphic in a punctured disc $D_r(z_0)\setminus\{z_0\}$.
  Suppose also that
  \[
    |f(z)|\leq A|z-z_0|^{-1+\epsilon}
  \]
  for some $\epsilon>0$, and all $z$ near $z_0$. Show that the singularity
  of $f$ at $z_0$ is removable.
\end{problem}

\begin{proof}
  We show this by proving the contrapositive. Suppose $z_0$ is a singularity
  that is not removable and let $\epsilon>0$ and $A\in\bb{R}_{>0}$.
  Then it is either a pole of order $k$ or essential.

  If $z_0$ is a pole of order $k$ then we can write $f$ as
  $f(z)=\sum_{n=0}^k\frac{a_{-n}}{(z-z_0)^{n}}\cdot g(z)$ where $g$ is holomorphic on
  $D_r(z_0)$. Then there exists a constant $B$ such that $B|(z-z_0)^{-k}|<|f(z)|$
  when $z$ is sufficiently close to $z_0$. In addition when $z$ is sufficiently
  close to $z_0$ we that $|f(z)|\geq B|z-z_0|^{-k}\geq A|z-z_0|^{-1+\epsilon}$ as $k\geq 1$.

  On the other hand if $k$ is an essential singularity we have a Laurant
  series in $D_r(z_0)\setminus\{z_0\}$ of the form
  \[
    f(z)=\sum_0^\infty b_n(z-z_0)^{-n} +\sum_0^\infty a_n(z-z_0)^n
  \]
  If we cut off the series for the negative powers at the first nonzero coefficient
  we get
  \[
    g(z)=b_k(z-z_0)^k+\sum_0^\infty a_n(z-z_0)^n
  \]
  where $|g|<|f|$ sufficiently close to $z_0$. Since $g$ has a pole of order $k$
  at $z_0$ this then reduces to the case where we have a pole of order $k$ where
  we shrink the radius of the disk such that $|g|<|f|$ holds within.
\end{proof}

\sk

\begin{problem}[3.9.3]
  If $f$ is holomorphic in the deleted neighborhood $\{0<|z-z_0|<r\}$ and has a
  pole of order $k$ at $z_0$, then we can write
  \[
    f(z)=\frac{a_{-k}}{(z-z_0)^k}+\cdots + \frac{a_{-1}}{(z-z_0)} +g(z)
  \]
  where $g$ is holomorphic in the disc $\{|z-z_0|<r\}$.
\end{problem}

\begin{proof}
  Since $f$ has a pole of order $k$ we can write $f$ as
  \[
    f(z)=(z-z_0)^k g(z)
  \]
  where $g$ is holomorphic. Moreover since $g$ is holomorphic it is equal
  to its power series $g(z)=\sum_0^\infty a_n(z-z_0)^n$. Then we distribute to get
  \[
    f(z)=\sum_{n=0}^{k-1}\frac{a_n}{(z-z_0)^{k-n}}+\sum_0^\infty a_{n+k}(z-z_0)^n
  \]
  completing the proof.
\end{proof}

%%%%%%%%%%%%%%%%%%%%%%%%%%%%%%%%%%%%%%%%%%%%%%%%%%%%%%%%%%%%%%%%%%%%%%%%%%%%%
\end{document}
