\documentclass[10pt]{article}
\usepackage[utf8]{inputenc}
\usepackage{amscd}
\usepackage{amsmath}
\usepackage{amssymb}
\usepackage{amsthm}
\usepackage{listings}
\usepackage{enumerate}
\usepackage[all,cmtip]{xy}

\textwidth=15cm \textheight=22cm \topmargin=0.5cm \oddsidemargin=0.5cm \evensidemargin=0.5cm

\newcommand{\sk}{\vskip 10mm}
\newcommand{\bb}[1]{\mathbb{#1}}
\newcommand{\ra}{\rightarrow}
\newcommand{\conj}[1]{\overline{#1}}
\DeclareMathOperator{\cis}{cis}

\theoremstyle{plain}
\newtheorem{problem}{Problem}
\newtheorem{lemma}{Lemma}[problem]

\theoremstyle{remark}
\newtheorem{tpart}{}[problem]
\newtheorem*{ppart}{}

\begin{document}

\begin{problem}[1.4.7]
  The family of mappings introduced here plays an important role in complex
  analysis. These mappings, sometimes called \textbf{Blaschke factors},
  will reappear in various applications in later chapters.
  \begin{enumerate}
  \item[(a)] Let $z,w$ be two complex numbers such that $\conj{z}w\neq 1$.
    Prove that
    \[
      \left|
        \frac{w-z}{1-\conj{w}z}
      \right|
      <1 \text{\quad if $|z|<1$ and $|w| <1$,}
    \]
    and also that
    \[
      \left|
        \frac{w-z}{1-\conj{w}z}
      \right|
      = 1 \text{\quad if $|z|=1$ or $|w|=1$}.
    \]
    Hint: Why can one assume that $z$ is real? It then suffices to prove that
    \[
      (r-w)(r-\conj{w})\leq (1-rw)(1-r\conj{w})
    \]
    with equality for appropriate $r$ and $|w|$.
  \item[(b)] Prove that for a fixed $w$ in the unit disc $\bb{D}$, the mappings
    \[
      F:z\mapsto \frac{w-z}{1-\conj{w}z}
    \]
    satisfies the following conditions:
    \begin{enumerate}
    \item[(i)] $F$ maps the unit disc to itself (that is,
      $F:\bb{D}\rightarrow\bb{D}$), and is holomorphic.
    \item[(ii)] $F$ interchanges $0$ and $w$, namely $F(0)=w$ and $F(w)=0$.
    \item[(iii)] $|F(z)|=1$ if $|z|=1$.
    \item[(iv)] $F:\bb{D}\rightarrow \bb{D}$ is bijective.
      Hint: Calculate $F\circ F$.
    \end{enumerate}
  \end{enumerate}
\end{problem}

\begin{proof}
  \begin{itemize}
  \item[(a)] Since $\conj{z}w\neq 1$ we can change the inequality from
    \[
      \left|
        \frac{w-z}{1-\conj{w}{z}}
      \right| \leq 1
    \]
    to
    \[
      |w-z|\leq |1-\conj{w}z|
    \]
    Then if we square both sides we can use the properties of the conjugate
    to get
    \[
      (w-z)\conj{(w-z)} \leq (1-\conj{w}z)\conj{(1-\conj{w}z)}
    \]
    Distribute the conjugate over the sum to get
    \[
      (w-z)(\conj{w}-\conj{z})\leq (1-\conj{w}z)(1-w\conj{z})
    \]
    Multiply out to get
    \[
      |w|^2-z\conj{w}-w\conj{z} + |z|^2\leq 1-\conj{w}z-\conj{z}w+|w|^2|z|^2
    \]
    Shuffle everything to the right and we get
    \[
      0 \leq 1-|w|^2|z|^2 -|w|^2-|z|^2 = (1-|w|^2)(1-|z|^2) = (1-|w|)(1+|w|)(1-|z|)(1+|z|)
    \]
    Since each one of these steps was invertible this inequality is equivalent
    to the original. At this point it is clear if either $|w|$ or $|z|$ is
    equal to one we have equality. Moreover when $|z|<1$ and $|w|<1$ we will
    have two negative terms and to positive terms making the inequality hold
    strictly.
  \item[(b)]
    \begin{itemize}
    \item[(i)] If $|z|\leq 1$ then  it is in the unit disc. From the above
      inequality $|F(z)|\leq 1$. Thus $F$ is a map from the unit disc to
      itself.

      For holomorphicity we multiply on the top and bottom by the conjugate of the bottom and
      rewrite with $w=a+bi$ and $z=x+iy$ to get:
      \[
        \frac{w-w^2\conj{z}-z+w|z|^2}{|1-\conj{w}z|^2}=
      \]
      \[
        \frac{a+bi+b^2x-a^2x-2iabx-b^2iy+a^2iy-2aby-x-iy+x^2a+y^2a+ibx^2+iby^2}{{\left(a x + b y - 1\right)}^{2} + {\left(b x - a y\right)}^{2}}
      \]

      The real part is
      \[
        u=\frac{a x^{2}+ay^2-ax-by+a-x}{{\left(a x + b y - 1\right)}^{2} + {\left(b x - a y\right)}^{2}}
      \]
      and the imaginary part is
      \[
        v=\frac{bx^2+by^2-bx+ay+b-y}{{\left(a x + b y - 1\right)}^{2} + {\left(b x - a y\right)}^{2}}
      \]
      % u = (a*x^2+a*y^2-a*x-b*y+a-x)/((a*x+b*y-1)^2+(b*x-a*y)^2)
      % v = (b*x^2+b*y^2-b*x+a*y+b-y)/((a*x+b*y-1)^2+(b*x-a*y)^2)
      Then we have
      \begin{align*}
        u_x &= \frac{2 \, {\left(b x^{2} + b y^{2} - b x + a y + b - y\right)} {\left({\left(b x - a y\right)} a - {\left(a x + b y - 1\right)} b\right)}}{{\left({\left(a x + b y - 1\right)}^{2} + {\left(b x - a y\right)}^{2}\right)}^{2}} + \frac{2 \, b y + a - 1}{{\left(a x + b y - 1\right)}^{2} + {\left(b x - a y\right)}^{2}}\\
        u_y &= -\frac{2 \, {\left(2 \, b y^{2} + a y - b y + b - y\right)} {\left({\left(a y - b y\right)} a + {\left(a y + b y - 1\right)} b\right)}}{{\left({\left(a y + b y - 1\right)}^{2} + {\left(a y - b y\right)}^{2}\right)}^{2}} + \frac{2 \, b y + a - 1}{{\left(a y + b y - 1\right)}^{2} + {\left(a y - b y\right)}^{2}}\\
        v_x &= \frac{2 \, {\left(b x^{2} + b y^{2} - b x + a y + b - y\right)} {\left({\left(b x - a y\right)} a - {\left(a x + b y - 1\right)} b\right)}}{{\left({\left(a x + b y - 1\right)}^{2} + {\left(b x - a y\right)}^{2}\right)}^{2}} + \frac{2 \, b y + a - 1}{{\left(a x + b y - 1\right)}^{2} + {\left(b x - a y\right)}^{2}}\\
        v_y &= -\frac{2 \, {\left(2 \, b y^{2} + a y - b y + b - y\right)} {\left({\left(a y - b y\right)} a + {\left(a y + b y - 1\right)} b\right)}}{{\left({\left(a y + b y - 1\right)}^{2} + {\left(a y - b y\right)}^{2}\right)}^{2}} + \frac{2 \, b y + a - 1}{{\left(a y + b y - 1\right)}^{2} + {\left(a y - b y\right)}^{2}}\\
      \end{align*}
    \item[(ii)] First calculate $F(0)$
      \begin{align*}
        F(0) &= \frac{w-0}{1-\conj{w}0}\\
             &= \frac{w}{1}\\
             &= w
      \end{align*}
      Calculating $F(w)$ we get
      \begin{align*}
        F(w) &= \frac{w-w}{1-|w|^2}\\
             &= 0
      \end{align*}
    \item[(iii)] From the second part of (a) if $|z|=1$ then $F(z)=1$.
    \item[(iv)] To show it is a bijection we calculate $F\circ F(z)$:
      \begin{align*}
        F\circ F(z) &= \frac{w+\frac{z-w}{1-\conj{w}z}}{1+\conj{w}\frac{z-w}{1-\conj{w}z}}\\
                &= \frac{\frac{w(1-\conj{w}z)+z-w}{1-\conj{w}z}}{\frac{1-\conj{w}z+\conj{z}-\conj{w}w}{1-\conj{w}z}}\\
                &= \frac{-|w|^2z+z}{1-|w|^2}\\
                &= \frac{z(1-|w|^2)}{1-|w|^2}\\
                &= z
      \end{align*}
    \end{itemize}
  \end{itemize}
\end{proof}

\sk

\begin{problem}[1.4.9]
  Show that in polar coordinates, the Cauchy-Riemann equations take the form
  \[
    \frac{\partial u}{\partial r}=\frac{1}{r}\frac{\partial v}{\partial\theta} \text{\quad and\quad} \frac{1}{r}\frac{\partial u}{\partial\theta}=-\frac{\partial v}{\partial r}.
  \]
  Use these equations to show that the logarithm function defined by
  \[
    \log z = \log r + i\theta \text{\quad  where $z=re^{i\theta}$ with $-\pi<\theta<\pi$}
  \]
  is holomorphic in the region $r>0$ and $-\pi<\theta<\pi$.
\end{problem}

\begin{proof}
  Using the chain rule and relations $x=r\cos\theta$ and $y=r\sin\theta$ we get:
  \begin{align*}
    u_r &= u_x\cos\theta+u_y\sin\theta\\
    u_\theta &= -u_xr\sin\theta+u_yr\cos\theta\\
    v_r &= v_x\cos\theta+v_y\sin\theta\\
    v_\theta &= -v_xr\sin\theta+v_yr\cos\theta\\
  \end{align*}

  At this point we have
  \[
    \frac{1}{r}(v_\theta)=-v_x\sin\theta+v_y\cos\theta=u_y\sin\theta+u_x\cos\theta=u_r
  \]
  and similarly
  \[
    \frac{1}{r}u_\theta = -u_x\sin\theta+u_y\cos\theta = -v_y\sin\theta-v_x\cos\theta=-v_r
  \]
\end{proof}

Taking the derivatives for $u=\log r$ and $v=\theta$ with respect to $r$ and $\theta$ we get
\begin{align*}
  \frac{\partial u}{\partial r} &= \frac{1}{r}\\
  \frac{\partial u}{\partial\theta} &= 0\\
  \frac{\partial v}{\partial r} &= 0\\
  \frac{\partial v}{\partial\theta} &= 1\\
\end{align*}

Which fulfill the polar Cauchy-Riemann equations. Thus $\log z$ is in fact
holomorphic.

\sk

\begin{problem}[1.4.10]
  Show that
  \[
    4\frac{\partial}{\partial z}\frac{\partial}{\partial \conj{z}}=4\frac{\partial}{\partial\conj{z}}\frac{\partial}{\partial z}=\Delta,
  \]
  where $\Delta$ is the \textbf{Laplacian}
  \[
    \Delta = \frac{\partial^2}{\partial x^2}+\frac{\partial^2}{\partial y^2}.
  \]
\end{problem}

\begin{proof}
  First recall that
  \[
    f_z=1/2f_x-i/2f_y
  \]
  and
  \[
    f_{\conj{z}}=1/2f_x+i/2f_y
  \]
  Then we compute:
  \begin{align*}
    4\frac{\partial}{\partial z}\frac{\partial}{\partial\conj{z}} &= 4\frac{\partial}{\partial z}(1/2f_x+i/2f_y)\\
                                                           &= 4(1/4f_{xx}+i/4f_{yx}-i/4f_{xy}+1/4f_{yy})\\
                                                           &= f_{xx}+f_{yy} = \Delta f
  \end{align*}
  and the other
  \begin{align*}
    4\frac{\partial}{\partial \conj{z}}\frac{\partial}{\partial z} &= 4\frac{\partial}{\partial \conj{z}}(1/2f_x-i/2f_y)\\
                                                           &= 4(1/4f_{xx}+i/4f_{yx}-i/4f_{xy}+1/4f_{yy})\\
                                                           &= f_{xx}+f_{yy} = \Delta f
  \end{align*}
\end{proof}

\sk

\begin{problem}[1.4.13]
  Suppose that $f$ is holomorphic in an open set $\Omega$. Prove that in any one
  of the following cases:
  \begin{itemize}
  \item[(a)] $\Re(f)$ is constant;
  \item[(b)] $\Im(f)$ is constant;
  \item[(c)] $|f|$ is constant;
  \end{itemize}
  one can conclude that $f$ is constant.
\end{problem}

\begin{proof}
  Separate $f=u+iv$.
  \begin{itemize}
  \item[(a)] Since $u$ is constant $u_x=u_y=0$. However using the
    Cauchy-Riemann equations we have that $v_x=v_y=0$ as well. Since
    $u$ and $v$ are constant so is $f$.
  \item[(b)] Since $v$ is constant $v_x=v_y=0$. However using the
    Cauchy-Riemann equations we have that $u_x=u_y=0$ as well. Since
    $u$ and $v$ are constant so is $f$.
  \item[(c)] Since $|f|$ is constant so is $|f|^2=u^2+v^2$. Take the
    derivative with respect to $x$ and divide by two to get
    \[
      uu_x+vv_x=0
    \]
    and for $y$
    \[
      uu_y+vv_y=0
    \]
    Next rewrite the latter with the Cauchy-Riemann equations
    \[
      -uv_x+vu_x=0
    \]
    Then multiply the first equation by $u$, the latter by $v$,
    and add them together to get
    \[
      u(uu_x+vv_y)v(-uv_x+vu_x)=(u^2+v^2)u_x=0
    \]
    Since $u^2+v^2$ is a constant this implies that $u_x=v_y=0$.

    We can do the same thing by rewriting the first to get
    \[
      (u^2+v^2)v_x=0
    \]
    Which implies that $v_x=-u_y=0$. This gives us that
    $u$ and $v$ are constant and as such so is $f$.
  \end{itemize}
\end{proof}

\sk

\begin{problem}[1.4.17]
  Show that if $\{a_n\}_{n=0}^\infty$ is a sequence of non-zero complex numbers
  such that
  \[
    \lim_{n\rightarrow\infty}\frac{|a_{n+1}|}{|a_n|}=L,
  \]
  then
  \[
    \lim_{n\rightarrow\infty}|a_n|^{1/n}=L.
  \]
  In particular, this exercise shows that when applicable, the ratio test
  can be used to calculate the radius of convergence of a power series.
\end{problem}

We prove the alternate exercise given in class.

\begin{proof}
  Let $R:=\lim_{n\rightarrow\infty}\frac{|a_n|}{|a_{n+1}|}$. We will show
  that $R$ is the radius of convergence for the power series
  $\sum_0^\infty a_n(z-z_0)$.

  Let $|z-z_0|<R$ and $r:=|z-z_0|$
  \[
    \left|
      \sum_0^\infty a_n(z-z_0)^n
    \right|
    \leq \sum_0^\infty |a_n|r^n
  \]

  Then we apply the ratio test to the latter sum to get
  \[
    \lim_{n\rightarrow\infty}\frac{|a_{n+1}|r}{|a_n|}=\frac{r}{R}<1
  \]
  which shows that the sum converges.

  Now suppose that $|z-z_0|>R$. Then apply the ratio test to
  the power series to get
  \[
    \lim_{n\rightarrow\infty}\frac{|a_{n+1}|r}{|a_n|} = \frac{r}{R}>1
  \]
  which shows that the series diverges.

  Therefore $R$ is in fact the radius of convergence.
\end{proof}

I suppose a way of proving the original would be to use the sequence
as a power series and apply the convergence tests with Hadarmard's rule.

\sk

\begin{problem}[2.6.1]
  Prove that
  \[
    \int_0^\infty\sin(x^2)dx=\int_0^\infty\cos(x^2)dx=\frac{\sqrt{2\pi}}{4}.
  \]
  These are the \textbf{Fresnel Integrals}. Here, $\int_0^\infty$ is
  interpreted as $\lim_{R\rightarrow\infty}\int_0^R$.\\
  Hint: Integrate the function $e^{-x^2}$ over the $\pi/4$ semicircle
  thing. Recall that $\int_{-\infty}^\infty e^{-x^2}=\sqrt{\pi}$.
\end{problem}

\begin{proof}
  Let $\gamma_R$ denote the curve given in the supplied figure of radius $R$.
  Then parametrize the curve piecewise so we have:
  \[
    \oint_{\gamma_R}e^{-iz^2} = \int_0^Re^{it^2}dt+\int_0^{\pi/4}iR\cis\theta e^{i(R\cis\theta)^2}d\theta +\int_0^R-\cis\pi/4e^{\cis(\pi/4)(R-t))^2 } dt
  \]
  The integral about $\gamma_R$ is zero as the function contained within is holomorphic.
  Now we will show that
  \[
    \lim_{R\rightarrow\infty}\int_0^{\pi/4}iR\cis\theta e^{i(R\cis\theta)^2}d\theta=0
  \]

  \begin{align*}
    \left|\int_0^{\pi/4}iR\cis\theta e^{iR^2\cis^2\theta}d\theta\right| &\leq \int_0^{\pi/4}R\left|e^{iR^2\cis2\theta}\right|d\theta\\
                                                              &= \int_0^{\pi/4}R\left|e^{iR^2\cos2\theta}\right|\left|e^{-R^2\sin2\theta}\right|d\theta\\
                                                              &\leq \int_0^{\pi/4}R|\cis(R^2\cos2\theta)|e^{-R^2\sin2\theta}d\theta\\
                                                              &\leq \int_0^{\pi/4}Re^{-R^2\sin2\theta}d\theta\\
  \end{align*}
  Next we use the fact that $\sin2\theta\geq\theta$ when $0\leq\theta\leq\pi/4$ to get
  \begin{align*}
    \int_0^{\pi/4}Re^{-R^2\sin2\theta}d\theta &\leq \int_0^{\pi/4}Re^{-R^2\theta}\\
                                         &=\left(\frac{Re^{-R^2\theta}}{-R^2}\right|_0^{\pi/4}\\
                                         &=\frac{e^{-R^2\pi/4}-1}{R}\\
  \end{align*}
  This last term goes to zero as $R$ approaches infinity.

  Next we evaluate $\int_0^R-\cis(\pi/4)e^{i(R-t)^2\cis(\pi/2)}dt$.
  \begin{align*}
    \int_0^R-\cis(\pi/4)e^{i(R-t)^2\cis^2(\pi/4)} dt &= \int_0^R-\cis(\pi/4)e^{i(R-t)^2\cis(\pi/2)}dt\\
                                              &= \int_0^R-\cis(\pi/4)e^{-(R-t)^2)}dt\\
  \end{align*}
  Then we apply the reminder in the problem giving that:
  \[
    \int_0^R-\cis(\pi/4)e^{-(R-t)^2)}dt = \cis(\pi/4)\frac{\sqrt{\pi}}{2}=-\left(\frac{\sqrt{2\pi}}{4}+i\frac{\sqrt{2\pi}}{4}\right)
  \]

  Which gives us
  \[
    \lim_{R\rightarrow\infty}\int_0^R e^{it^2}dt=\lim_{R\rightarrow\infty}\int_0^R\cos t^2+i\sin t^2 dt = \left(\frac{\sqrt{2\pi}}{4}+i\frac{\sqrt{2\pi}}{4}\right)
  \]

  If we take the real part of the above equation we get
  \[
    \int_0^\infty\cos(x^2)dx=\frac{\sqrt{2\pi}}{4}
  \]
  and if we take the imaginary part we get
  \[
    \int_0^\infty\sin(x^2)dx=\frac{\sqrt{2\pi}}{4}
  \]
\end{proof}

\sk

\begin{problem}[2.6.11]
  Let $f$ be a holomorphic function on the disc $D_{R_0}$ centered at the
  origin and of radius $R_0$.
\end{problem}

\begin{proof}
  \begin{itemize}
  \item[(a)] Prove that whenever $0<R<R_0$ and $|z|<R$, then
    \[
      f(z)=\frac{1}{2\pi}\int_0^{2\pi}f(Re^{i\varphi}\Re\left(\frac{Re^{i\varphi}+z}{Re^{i\varphi}-z}\right))d\varphi.
    \]
  \item[(b)] Show that
    \[
      \Re\left(\frac{Re^{i\gamma}-r}{Re^{i\gamma}-r}\right) = \frac{R^2-r^2}{R^2-2Rr\cos\gamma + r^2}.
    \]
  \end{itemize}
  Hint: For the first part, note that if $w=R^2/\conj{z}$, then the integral of
  $f(\zeta)/(\zeta-w)$ around the circle of radius $R$ centered at the origin is
  zero. Use this, together with the usual Cauchy integral formula, to deduce
  the desired identity.
\end{proof}

%%%%%%%%%%%%%%%%%%%%%%%%%%%%%%%%%%%%%%%%%%%%%%%%%%%%%%%%%%%%%%%%%%%%%%%%%%%%%
\end{document}
