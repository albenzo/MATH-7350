\documentclass[10pt]{article}
\usepackage[utf8]{inputenc}
\usepackage{amscd}
\usepackage{amsmath}
\usepackage{amssymb}
\usepackage{amsthm}
\usepackage{listings}
\usepackage{enumerate}
\usepackage[all,cmtip]{xy}

\textwidth=15cm \textheight=22cm \topmargin=0.5cm \oddsidemargin=0.5cm \evensidemargin=0.5cm

\newcommand{\sk}{\vskip 10mm}
\newcommand{\bb}[1]{\mathbb{#1}}
\newcommand{\ra}{\rightarrow}
%\newcommand{\Re}{\text{Re}}
%\newcommand{\Im}{\text{Im}}

\theoremstyle{plain}
\newtheorem{problem}{Problem}
\newtheorem{lemma}{Lemma}[problem]

\theoremstyle{remark}
\newtheorem{tpart}{}[problem]
\newtheorem*{ppart}{}

\begin{document}

\begin{problem}[1.4.7]
  The family of mappings introduced here plays an important role in complex
  analysis. These mappings, sometimes called \textbf{Blaschke factors},
  will reappear in various applications in later chapters.
  \begin{enumerate}
  \item[(a)] Let $z,w$ be two complex numbers such that $\bar{z}w\neq 1$.
    Prove that
    \[
      \left|
        \frac{w-z}{1-\bar{w}z}
      \right|
      <1 \text{\quad if $|z|<1$ and $|w| <1$,}
    \]
    and also that
    \[
      \left|
        \frac{w-z}{1-\bar{w}z}
      \right|
      = 1 \text{\quad if $|z|=1$ or $|w|=1$}.
    \]
    Hint: Wy can one assume that $z$ is real? It then suffices to prove that
    \[
      (r-w)(r-\bar{w})\leq (1-rw)(1-r\bar{w})
    \]
    with equality for appropriate $r$ and $|w|$.
  \item[(b)] Prove that for a fixed $w$ in the unit disc $\bb{D}$, the mappings
    \[
      F:z\mapsto \frac{w-z}{1-\bar{w}z}
    \]
    satisfies the following conditions:
    \begin{enumerate}
    \item[(i)] $F$ maps the unit disc to itself (that is,
      $F:\bb{D}\rightarrow\bb{D}$), and is holomorphic.
    \item[(ii)] $F$ interchanges $0$ and $w$, namely $F(0)=w$ and $F(w)=0$.
    \item[(iii)] $|F(z)|=1$ if $|z|=1$.
    \item[(iv)] $F:\bb{D}\rightarrow \bb{D}$ is bijective.
      Hint: Calculate $F\circ F$.
    \end{enumerate}
  \end{enumerate}
\end{problem}

\begin{proof}
  
\end{proof}

\sk

\begin{problem}[1.4.9]
  Show that in polar coordinates, the Cauchy-Riemann equations take the form
  \[
    \frac{\partial u}{\partial r}=\frac{1}{r}\frac{\partial v}{\partial\theta} \text{\quad and\quad} \frac{1}{r}\frac{\partial u}{\partial\theta}=-\frac{\partial v}{\partial r}.
  \]
  Use these equations to show that the logarithm function defined by
  \[
    \log z = \log r + i\theta \text{\quad  where $z=re^{i\theta}$ with $-\pi<\theta<\pi$}
  \]
  is holomorphic in the region $r>0$ and $-\pi<\theta<\pi$.
\end{problem}

\begin{proof}
  
\end{proof}

\sk

\begin{problem}[1.4.10]
  Show that
  \[
    4\frac{\partial}{\partial z}\frac{\partial}{\partial \bar{z}}=4\frac{\partial}{\partial\bar{z}}\frac{\partial}{\partial z}=\Delta,
  \]
  where $\Delta$ is the \textbf{Laplacian}
  \[
    \Delta = \frac{\partial^2}{\partial x^2}+\frac{\partial^2}{\partial y^2}.
  \]
\end{problem}

\begin{proof}
  
\end{proof}

\sk

\begin{problem}[1.4.13]
  Suppose that $f$ is holomorphic in an open set $\Omega$. Prove that in any one
  of the following cases:
  \begin{itemize}
  \item[(a)] $\Re(f)$ is constant;
  \item[(b)] $\Im(f)$ is constant;
  \item[(c)] $|f|$ is constant;
  \end{itemize}
  one can conclude that $f$ is constant.
\end{problem}

\begin{proof}
  
\end{proof}

\sk

\begin{problem}[1.4.17]
  Show that if $\{a_n\}_{n=0}^\infty$ is a sequence of non-zero complex numbers
  such that
  \[
    \lim_{n\rightarrow\infty}\frac{|a_{n+1}|}{|a_n|}=L,
  \]
  then
  \[
    \lim_{n\rightarrow\infty}|a_n|^{1/n}=L.
  \]
  In particular, this exercise shows that when applicable, the ratio test
  can be used to calculate the radius of convergence of a power series.
\end{problem}

\begin{proof}
  
\end{proof}

\sk

\begin{problem}[2.6.1]
  Prove that
  \[
    \int_0^\infty\sin(x^2)dx=\int_0^\infty\cos(x^2)dx=\frac{\sqrt{2\pi}}{4}.
  \]
  These are the \textbf{Fresnel Integrals}. Here, $\int_0^\infty$ is
  interpreted as $\lim_{R\rightarrow\infty}\int_0^R$.\\
  Hint: Integrate the function $e^{-x^2}$ over the $\pi/4$ semicircle
  thing. Recall that $\int_{-\infty}^\infty e^{-x^2}=\sqrt{\pi}$.
\end{problem}

\begin{proof}
  
\end{proof}

\sk

\begin{problem}[2.6.11]
  Let $f$ be a holomorphic function on the disc $D_{R_0}$ centered at the
  origin and of radius $R_0$.
\end{problem}

\begin{proof}
  \begin{itemize}
  \item[(a)] Prove that whenever $0<R<R_0$ and $|z|<R$, then
    \[
      f(z)=\frac{1}{2\pi}\int_0^{2\pi}f(Re^{i\varphi}\Re\left(\frac{Re^{i\varphi}+z}{Re^{i\varphi}-z}\right))d\varphi.
    \]
  \item[(b)] Show that
    \[
      \Re\left(\frac{Re^{i\gamma}-r}{Re^{i\gamma}-r}\right) = \frac{R^2-r^2}{R^2-2Rr\cos\gamma + r^2}.
    \]
  \end{itemize}
  Hint: For the first part, note that if $w=R^2/\bar{z}$, then the integral of
  $f(\zeta)/(\zeta-w)$ around the circle of radius $R$ centered at the origin is
  zero. Use this, together with the usual Cauchy integral formula, to deduce
  the desired identity.
\end{proof}

\sk

\begin{problem}[2.6.14]
  Suppose that $f$ is holomorphic in an open set containing the closed unit
  disc, except for a pole at $z_0$ on the unit circle. Show that if
  \[
    \sum_{n=0}^\infty a_nz^n
  \]
  denotes the power series expansion of $f$ in the open unit disc, then
  \[
    \lim_{n\rightarrow\infty}\frac{a_n}{a_{n+1}}=z_0
  \]
\end{problem}

\begin{proof}
  
\end{proof}

%%%%%%%%%%%%%%%%%%%%%%%%%%%%%%%%%%%%%%%%%%%%%%%%%%%%%%%%%%%%%%%%%%%%%%%%%%%%%
\end{document}
