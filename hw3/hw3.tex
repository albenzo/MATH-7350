\documentclass[10pt]{article}
\usepackage[utf8]{inputenc}
\usepackage{amscd}
\usepackage{amsmath}
\usepackage{amssymb}
\usepackage{amsthm}
\usepackage{listings}
\usepackage{enumerate}
\usepackage[all,cmtip]{xy}

\textwidth=15cm \textheight=22cm \topmargin=0.5cm \oddsidemargin=0.5cm \evensidemargin=0.5cm

\newcommand{\sk}{\vskip 10mm}
\newcommand{\bb}[1]{\mathbb{#1}}
\newcommand{\ra}{\rightarrow}
\newcommand{\conj}[1]{\overline{#1}}
\newcommand{\wt}[1]{\widetilde{#1}}
\DeclareMathOperator{\cis}{cis}
\DeclareMathOperator{\res}{Res}


\theoremstyle{plain}
\newtheorem{problem}{Problem}
\newtheorem{lemma}{Lemma}[problem]

\theoremstyle{remark}
\newtheorem{tpart}{}[problem]
\newtheorem*{ppart}{}

\begin{document}

\begin{problem}[]
  Consider the integral
  \[
    \Phi_n(z) = \int_a^b \frac{x^ndx}{(x-a)^{\frac{1}{2}-\lambda} (b-x)^{\frac{1}{2}+ \lambda} (x-z)}
  \]
  where $z \notin [a,b]$ and $-\frac{1}{2} < \lambda < \frac{1}{2}$.
  
  \begin{enumerate}
  \item[(a)] Compute the integral for $n= 0$ and $n=1$.
  \item[(b)] Determine the domain where $\phi_n(z)$ is holomorphic.
  \end{enumerate}
  
  Hint: Consider the contour $\Gamma$ (circle entering to go about branch points) and fix a
  branch of the function $\Gamma = C_R \cup \gamma^- \cup \gamma^+$ and
  $f(\zeta) = (\zeta - a)^{\frac{1}{2} - \lambda}(b- \zeta) ^{\frac{1}{2} + \lambda}$
  where $\zeta = x+ iy$
\end{problem}

\begin{enumerate}
\item[(a)] Let $f(w)=(w-a)^{1/2-\lambda}(b-w)^{1/2+\lambda}$ with $w=s+iy$. Then let $\wt{f}(w)=f_a(w)f_b(w)$
  where
  \[
    f_a(w)=|w-a|^{1/2-\lambda}e^{i\theta_a(1/2-\lambda)},\quad f_b(w)=|b-w|^{1/2+\lambda}e^{i\theta_b(1/2+\lambda)}
  \]
  with $\theta_a=\arg(w-a)$ and $\theta_b=\arg(b-w)$. Using the picture I will draw here

  \vskip2in

  We do the work for finding branch cuts and the like to make it all consistent.
  We have $a,b,\infty$ as branch points with a branch cut along $[a,b]$.
  \begin{align*}
    z=s+i0 & s>b & \theta_a=0\ & \theta_b =-\pi & \wt{f}(s+i0)=-i|\wt{f}(s)|e^{-i\pi\lambda}\\
    z=s+i0 & a\leq s\leq b & \theta_a=0\ & \theta_b =0 & \wt{f}(s)=|\wt{f}(s)|\\
    z=s+i0 & s<a & \theta_a=0\ & \theta_b =-\pi & \wt{f}(s+i0)=i|\wt{f}(s)|e^{-i\pi\lambda}\\
    z=s-i0 & a\leq s\leq b & \theta_a=2\pi\ & \theta_b =0 & \wt{f}(s-i0)=-|\wt{f}(s)|e^{-2i\pi\lambda}\\
    z=s-i0 & b<s & \theta_a=2\pi\ & \theta_b =\pi & \wt{f}(s+i0)=-i|\wt{f}(s)|e^{-i\pi\lambda}\\
  \end{align*}
  
  Now we integrate along the suggested curve. Since the function has only a single
  simple pole at $z$ within the curve we have
  \begin{align*}
    \int_\Gamma \frac{w^n dw}{\wt{f}(w)(w-z)}&=\left(\int_{C_R}+\int_{\gamma_+}+\int_{\gamma_-}+\int_{C_{\epsilon,a^-}}+\int_{C_{\epsilon,b^+}}+\int_{C_{\epsilon,b^+}}+\int_{C_{\epsilon,b^-}}\right)\frac{w^n dw}{\wt{f}(w)(w-z)}\\
                                    &=2\pi i\res_{w=z}\frac{w^n }{\wt{f}(w)(w-z)}    
  \end{align*}

  We ignore the two fragments of the integral along $[b,R]$ since they cancel each other out.
  At this point the cases for $n=0$ and $n=1$ diverge.
  When $n=0$ it the integral along $C_R$ and the various $C_{\epsilon}$s as the top doesn't
  change, yet the bottom will grow at $R^2$ as $R\rightarrow\infty$. When $n=1$ I'm not entirely sure
  how to make them vanish as once we account for arclength the top and bottom appear
  to grow at the same rate.

  Regardless, as the pole is simple the residue is
  \[
    2\pi i\res_{w=z}\frac{w^n }{\wt{f}(w)(w-z)} = 2\pi i\frac{z^n}{\wt{f}(z)}
  \]

  The last bit is to combine the two integrals along $\gamma_+$ and $\gamma_-$. Using
  the values deduced earlier we have
  \[
    \int_{\gamma_+} \frac{w^n dw}{\wt{f}(w)(w-z)}+ \int_{\gamma_-}\frac{w^n dw}{\wt{f}(w)(w-z)} = \int_a^b \frac{w^n dw}{\wt{f}(s+i0)(w-z)}+ \int_b^a\frac{w^n dw}{\wt{f}(s-i0)(w-z)} =
    (1+e^{2\pi i \lambda})\int_a^b\frac{s^nds}{f(s)(s-z)}
  \]
  Putting it all together we get, for $n=0,1$ that
  \[
    \int_a^b\frac{s^nds}{f(s)(s-z)}=\frac{2\pi i z^n}{\wt{f}(z)(1+e^{2\pi i \lambda})}
  \]

\item[(b)] The inside of the integral is holomorphic on $\bb{C}\setminus x$ for fixed $x\in(a,b)$.
  Moreover it is continuous on $\bb{C}\setminus[a,b]\times(a,b)$. As such we have that $\Phi_n(z)$ is holomorphic
  on $\bb{C}\setminus[a,b]$.
\end{enumerate}

\sk

\begin{problem}[]
  Show, by contour integration, that if $a >0$ and $\xi \in \mathbb{R}$ then
  \[
    \frac{1}{\pi} \int_{-\infty}^\infty \frac{a}{a^2 + x^2}e^{-2\pi ix\xi}dx = e^{-2\pi a |\xi|},
  \]
  and check that 
  \[
    \int_{-\infty}^\infty e^{-2\pi a|\xi|}e^{2\pi i\xi x}d\xi = \frac{1}{\pi}\frac{a}{a^2 + x^2}.
  \]
\end{problem}

\begin{proof}
  We start with the first integral in the case where $\xi\geq 0$. In this scenario
  we integrate over the lower semicircle of radius $R$. The integral of the
  outer portion $C_{R^-}$ will approach zero by Jordan's lemma as
  $|\frac{a}{a^2+z^2}|\leq\frac{a}{R^2-a^2}\rightarrow 0 $ as $R\rightarrow\infty$. Thus we have
  \[
    \frac{1}{\pi}\int_\infty^{-\infty}\frac{a}{a^2 + x^2}e^{-2\pi ix\xi}dx = 2 i \res_{z=-ia}\frac{a}{a^2 + z^2}e^{-2\pi iz\xi}
  \]
  The residue is
  \[
    2i\res_{z=-ia}\frac{a}{a^2 + z^2}e^{-2\pi iz\xi}=2i\frac{a}{-2ia}e^{-2\pi i(-ia)|\xi|} = -e^{-2\pi a\xi}
  \]
  Which, after swapping the bounds of integration, implies that when $\xi\geq 0$
  \[
    \frac{1}{\pi} \int_{-\infty}^\infty \frac{a}{a^2 + x^2}e^{-2\pi ix\xi}dx = e^{-2\pi a |\xi|}
  \]

  However when $\xi<0$ we instead integrate over the upper semicircle of radius $R$. For the
  same reason as above the outer portion of the integral approaches 0 as $R\rightarrow\infty$. This gives
  us
  \[
    \frac{1}{\pi}\int_{-\infty}^I\frac{a}{a^2 + x^2}e^{-2\pi ix\xi} dx = 2 i \res_{z=ia}\frac{a}{a^2 + z^2}e^{-2\pi iz\xi}
  \]
  Similar to before if we do the residue calculation we get
  \[
    2 i \res_{z=ia}\frac{a}{a^2 + z^2}e^{-2\pi iz\xi}= 2 i \frac{a}{2ia}e^{2\pi a\xi}=e^{-2\pi a\xi}
  \]

  Therefore
  \[
    \frac{1}{\pi} \int_{-\infty}^\infty \frac{a}{a^2 + x^2}e^{-2\pi ix\xi}dx = e^{-2\pi a |\xi|},
  \]
  for all $\xi\in \bb{R}$.

  To check the other direction split the integral into two pieces. The first
  for positive numbers and the other the negatives.
  \[
    \int_{-\infty}^\infty e^{-2\pi a|\xi|}e^{2\pi i\xi x}d\xi = \int_0^\infty e^{-2\pi a|\xi|}e^{2\pi i\xi x}d\xi-\int_{0}^{-\infty} e^{-2\pi a|\xi|}e^{2\pi i\xi x}d\xi
  \]
  Substitute $\zeta=-\xi$ into the latter to get
  \begin{align*}
    \int_0^\infty e^{-2\pi a\xi}e^{2\pi i\xi x}d\xi-\int_{0}^{-\infty} e^{-2\pi a|\xi|}e^{2\pi i\xi x}d\xi &= \int_0^\infty e^{-2\pi a\xi}e^{2\pi i\xi x}d\xi+\int_{0}^{\infty} e^{-2\pi a\zeta}e^{-2\pi i\zeta x}d\zeta\\
                                                                                      &= \int_0^\infty e^{(-2\pi a+2\pi i x)\xi}d\xi+\int_{0}^{\infty} e^{(-2\pi a-2\pi ix)\zeta}d\zeta\\
                                                                                      &=\frac{1}{2\pi a-2i\pi x}+\frac{1}{2\pi a+2i\pi x}\\
                                                                                      &=\frac{1}{\pi}\frac{a}{a^2+x^2}
  \end{align*}
  Which completes the second part of the problem.
\end{proof}

\sk

\begin{problem}[]
  Prove that 
  \[
    \frac{1}{\pi} \sum_{n = -\infty}^\infty 
    \frac{a}{a^2 + n^2} = \sum_{n = -\infty}^\infty e^{-2\pi a|n|}
  \]
  whenever $a>0$. Hence show that the sum equals both $\coth \pi a$.
\end{problem}

\begin{proof}
  By the previous problem and the fact that $\frac{a}{a^2+x^2}\in\mathcal{F}$ the two
  above sums are equal by the Poisson summation formula. Now we show that this sum
  is equal to $\coth\pi a$.
  \begin{align*}
    \coth\pi a &= \frac{e^{\pi a}+e^{-\pi a}}{e^{\pi a}-e^{-\pi a}}\\
             &= \frac{e^{\pi a} (1+e^{-2\pi a})}{e^{\pi a}(1-e^{-2\pi a})}\\
             &= (1+e^{-2\pi a})\frac{1}{1-e^{-2\pi a}}\\
             &= (1+e^{-2\pi a})\sum_{n=0}^\infty e^{-2n\pi a}\\
             &= \sum_{n=0}^\infty e^{-2n\pi a}+\sum_{n=1}^\infty e^{-2 n\pi a}\\
             &= \sum_{n=-\infty}^\infty e^{-2|n|\pi a}
  \end{align*}
\end{proof}

\sk

\begin{problem}[]
  \begin{enumerate}
  \item[(a)] Let $\tau$ be fixed with $\Im(\tau)>0$. Apply the Poisson summation
    formula to
    \[
      f(z) = (\tau + z)^{-k},
    \]
    where $k$ is an integer $\geq 2$, to obtain
    \[
      \sum_{n = -\infty}^\infty\frac{1}{(\tau +n)^k} = \frac{(-2\pi i )^k}{(k-1)!} \sum_{m=1}^{\infty} m^{k-1}e^{2\pi im\tau}.
    \]
  \item[(b)] Set $k=2$ in the above formula to show that if $\Im(\tau)>0$, then 
    \[
      \sum_{n = -\infty}^\infty 
      \frac{1}{(\tau +n)^2} = \frac{\pi^2}{\sin^2(\pi\tau)}.
    \]
  \item[(c)] Can one conclude that the above formula hold true whenever $\tau$
    is any complex number that is not an integer?
  \end{enumerate}
\end{problem}

\begin{enumerate}
\item[(a)] Since $f\in\mathcal{F}$ when $k\geq 2$ we can use the Poisson
  summation formula to show the equivalence of these two series. The
  Fourier transform of $f$ is of the form
  \[
    \hat{f}(\xi)=\int_{-\infty}^\infty f(x)e^{-2\pi ix\xi}dx
  \]
  When $\xi<0$ we calculate the above integral by integrating over
  the upper semicircle of radius $R$. The outer portion $C_R$ will
  approach $0$ by Jordan's lemma as $|f(R)|\leq\frac{1}{(|R|-|\tau|)^k}$ approaches
  zero as $R\rightarrow\infty$. Since $f(z)e^{-2\pi i x\xi}$ is holomorphic in the upper half plane
  the integral over the upper semicircle is also zero. This implies that
  \[
    \hat{f}(\xi)=\int_{-\infty}^\infty f(z)e^{-2\pi iz\xi}dz = (\int_{C_R}+\int_{-R}^R)f(z)e^{-2\pi iz\xi}dz = \int_{-\infty}^\infty f(x)e^{-2\pi ix\xi} dx = 0
  \]
  when $\xi<0$.

  On the other hand if $\xi>0$, we integrate over the lower semicircle of radius
  $R$. The integral on $C_{R^-}$ will be zero by Jordan's lemma as before. This
  gives us
  \[
    \hat{f}(\xi)=\int_{-\infty}^\infty f(x)e^{-2\pi i x\xi}dx = -2\pi i\res_{x=-\tau}\frac{e^{-2\pi i x\xi}}{(\tau+z)^k}
  \]

  Next we calculate the residue
  \begin{align*}
    \hat{f}(\xi)=-2\pi i \res_{x=-\tau}\frac{e^{-2\pi i x\xi}}{(\tau+z)} &=-2\pi i\lim_{x\rightarrow-\tau}\frac{1}{(k-1)!}\frac{d^{k-1}}{dx^{k-1}}\frac{(\tau+x)^k e^{-2\pi i x\xi}}{(\tau+x)^k}\\
                                                          &=-2\pi i\lim_{x\rightarrow-\tau}\frac{1}{(k-1)!}\frac{d^{k-1}}{dx^{k-1}}e^{-2\pi i x\xi}\\
                                                          &=-2\pi i\lim_{x\rightarrow-\tau} \frac{(-2\pi i\xi)^{k-1}}{(k-1)!}e^{-2\pi i x\xi}\\
                                                          &=\frac{(-2\pi i)^{k}\xi^{k-1}}{(k-1)!}e^{2\pi i \tau\xi}\\
  \end{align*}
  Applying the Poisson summation formula with the above values for $\hat{f}(\xi)$
  gives us that
  \[
    \sum_{n = -\infty}^\infty\frac{1}{(\tau +n)^k} = \frac{(-2\pi i )^k}{(k-1)!} \sum_{m=1}^{\infty} m^{k-1}e^{2\pi im\tau}.
  \]
\item[(b)] First we transform $\frac{\pi^2}{\sin^2(\pi\tau)}$ into a series.
  \begin{align*}
    \frac{\pi^2}{\sin^2(\pi\tau)} &= \pi^2\left(\frac{2i}{e^{i\pi\tau}-e^{-i\pi\tau}}\right)^2\\
                               &=\frac{-4\pi^2}{e^{-i\pi\tau/2}(1-e^{2i\pi\tau})^2}\\
                               &=\frac{-4\pi^2}{e^{-2i\pi\tau}}\sum_{m=0}^\infty me^{2i\pi m\tau}\\
                               &=-4\pi^2\sum_{1}^\infty me^{2i\pi m\tau}\\
  \end{align*}
  The last step only works when $|e^{2\pi i\tau}|<1$. However since $\Im(\tau)>0$
  this is in fact the case.

  Which exactly matches the series listed above when we plug in $k=2$.
\item[(c)] This will not work if $\Im(\tau)>0$ does not hold. The series in
  question may not be convergent otherwise.
\end{enumerate}

\sk

\begin{problem}[]
  Compute the integral 
  \[
    I = \int_0^\infty \frac{\log^2(x)}{x^2 + a^2} dx
  \]
\end{problem}

Let $\Gamma$ be the positively oriented upper half annulus of outer radius $R$
and inner radius $\epsilon$. The given function has a simple pole at $ia$ inside $\Gamma$.
This gives us
\[
  \int_\epsilon^R\frac{\ln^2(x)}{x^2 + a^2} dx + \int_{C_R}\frac{\log^2(z)}{z^2 + a^2} dz+ \int_{C_\epsilon}\frac{\log^2(z)}{z^2 + a^2} dz+\int_{-R}^{-\epsilon}\frac{\log^2(z)}{z^2 + a^2} dz=2\pi i\res_{z=ia}\frac{\log^2(z)}{z^2 + a^2}
\]

The integrals about $C_R$ and $C_\epsilon$ both go to zero. For
\[
  \int_{-R}^{-\epsilon}\frac{\log^2(z)}{z^2 + a^2} dz
\]
we make the substitution $z=-x$ giving us
\[
  \int_{-R}^{-\epsilon}\frac{\log^2(z)}{z^2 + a^2} dz = \int_{\epsilon}^R\frac{\log^2(-x)}{x^2 + a^2} dx
\]
Which when rewriting $\log(-x)$ as $\ln(x)+i\pi$ we get
\[
  \int_{\epsilon}^R\frac{\log^2(-x)}{x^2 + a^2} dx = \int_{\epsilon}^R\frac{\ln^2(x)+\pi i\ln(x)-\pi^2}{x^2 + a^2} dx
\]

Now rewrite the original equation as
\[
  2\int_\epsilon^R\frac{\ln^2(x)}{x^2+a^2}dx = 2\pi i\res_{z=ia}\frac{\log^2(z)}{z^2 + a^2} +\pi^2\int_\epsilon^R\frac{1}{x^2+a^2}dx-2\pi i\int_\epsilon^R\frac{\ln(x)}{x^2+a^2}dx
\]
The residue is
\[
  \res_{z=ia}\frac{\log^2z}{z^2+a^2}=\frac{\log^2(ia)}{2ia}=\frac{\ln^2(a)+\pi i\ln(a)-\pi^2/4}{2ia}
\]

The first integral on the right is of the derivative of $a^{-1}\arctan(x/a)$. Giving us
\[
  \int_0^\infty\frac{1}{a^2+x^2}dx=\frac{\pi}{2a}
\]
The latter integral is from a previous assignment
\[
  \int_0^\infty\frac{\ln x}{x^2+a^2}dx = \frac{\pi\ln a}{2a}
\]

Putting this all together while letting $\epsilon\rightarrow 0$ and $R\rightarrow\infty$ we get
\begin{align*}
  2I &= 2\pi i\res_{z=ia}\frac{\log^2(z)}{z^2 + a^2} +\pi^2\int_\epsilon^R\frac{1}{x^2+a^2}dx-2\pi i\int_\epsilon^R\frac{\ln(x)}{x^2+a^2}dx\\
     &=\frac{\pi\ln^2 a}{a}+\frac{i\pi^2\ln a}{a}-\frac{\pi^3}{4a}+\frac{\pi^3}{2a}-\frac{i\pi^2\ln a}{a}\\
     &=\frac{\pi\ln^2 a}{a}-\frac{\pi^3}{4a}+\frac{\pi^3}{2a}\\
\end{align*}

We can then conclude that
\[
  I = \int_0^\infty \frac{\log^2 x}{x^2+a^2} dx = \frac{\pi\ln^2 a}{2a}-\frac{\pi^3}{8a}+\frac{\pi^3}{4a}
\]

%%%%%%%%%%%%%%%%%%%%%%%%%%%%%%%%%%%%%%%%%%%%%%%%%%%%%%%%%%%%%%%%%%%%%%%%%%%%% 
\end{document}
